\documentclass[12pt,a4paper]{book}
\usepackage[utf8x]{inputenc}   % omogoča uporabo slovenskih črk kodiranih v formatu UTF-8
\usepackage[slovene,english]{babel}    % naloži, med drugim, slovenske delilne vzorce
\usepackage[pdftex]{graphicx}  % omogoča vlaganje slik različnih formatov
\usepackage{fancyhdr}          % poskrbi, na primer, za glave strani
\usepackage{comment}
\usepackage[pdftex, colorlinks=true,
						citecolor=black, filecolor=black, 
						linkcolor=black, urlcolor=black,
						pagebackref=false, 
						pdfproducer={LaTeX}, pdfcreator={LaTeX}, hidelinks]{hyperref}
\usepackage{color}       % dodal Solina
%\usepackage{soul}
\usepackage{amsthm}
\usepackage{amsmath}
\usepackage{amsfonts}
\usepackage{amssymb}

\usepackage{amsmath}
\usepackage{mathtools}
\usepackage{enumitem}
\usepackage{commath}
\usepackage{array,xparse}

\usepackage{ulem} % ta knjižnica redefinira komando \emph
% ta komanda jo nastavi nazaj na privzeto vrednost
\normalem

\usepackage{tikz-cd} 
\usetikzlibrary{babel} % this fixes problems with tikz-cd
% http://tex.stackexchange.com/questions/166772/problem-with-babel-and-tikz-using-draw



%%%%%%%%%%%%%%%%%%%%%%%%%%%%%%%%%%%%%%%%
%	DIPLOMA INFO
%%%%%%%%%%%%%%%%%%%%%%%%%%%%%%%%%%%%%%%%
\newcommand{\ttitle}{Yonedova lema in njena uporaba}
\newcommand{\ttitleEn}{Naslov EN}
\newcommand{\tsubject}{\ttitle}
\newcommand{\tsubjectEn}{\ttitleEn}
\newcommand{\tauthor}{Jure Taslak}
\newcommand{\tkeywords}{računalnik, računalnik, računalnik}
\newcommand{\tkeywordsEn}{computer, computer, computer}


%%%%%%%%%%%%%%%%%%%%%%%%%%%%%%%%%%%%%%%%
% naslovi
%%%%%%%%%%%%%%%%%%%%%%%%%%%%%%%%%%%%%%%%  
\newcommand{\autfont}{\Large}
\newcommand{\titfont}{\LARGE\bf}
\newcommand{\clearemptydoublepage}{\newpage{\pagestyle{empty}\cleardoublepage}}
\setcounter{tocdepth}{1}



%%%%%%%%%%%%%%%%%%%%%%%%%%%%%%%%%%%%%
%									%
% konstrukti						%
%									%
%%%%%%%%%%%%%%%%%%%%%%%%%%%%%%%%%%%%%  

\theoremstyle{definition}
\newtheorem{definicija}{Definicija}[chapter]
 
\theoremstyle{plain}
\newtheorem{izrek}[definicija]{Izrek}
\newtheorem{trditev}[definicija]{Trditev}
\newtheorem{posledica}{Posledica}[definicija]
\newtheorem{lema}[definicija]{Lema}
\newenvironment{dokaz}{\emph{Dokaz.}\ }{\hspace{\fill}{$\Box$}}

\theoremstyle{definition}
\newtheorem{primer}{Primer}[section]
\newtheorem*{primer*}{Primer}

\theoremstyle{remark}
\newtheorem*{opomba}{Opomba}

%%%%%%%%%%%%%%%%%%%%%%%%%%%%%%%%%
%								%
%	Tikz-cd nastavitve			%
%								%
%%%%%%%%%%%%%%%%%%%%%%%%%%%%%%%%%

\tikzcdset{
 	diagrams={sep=large},
	labels={font=\small}
}

%%%%%%%%%%%%%%%%%%%%%%%%%%%%%
% Avtor in naslov			%
%%%%%%%%%%%%%%%%%%%%%%%%%%%%%

\author{Jure Taslak}
\title{Yonedova lema in njena uporaba}

%%%%%%%%%%%%%%%%%%%%%%%%%%%%%%%%%%%%%%%%
% pdfInfo
%%%%%%%%%%%%%%%%%%%%%%%%%%%%%%%%%%%%%%%%  
\pdfinfo{%
    /Title    (\ttitle)
    /Author   (\tauthor, damjan@cvetan.si)
    /Subject  (\ttitleEn)
    /Keywords (\tkeywordsEn)
    /ModDate  (\pdfcreationdate)
    /Trapped  /False
}

%%%%%%%%%%%%%%%%%%%%%%%%%%%%%%%%%%%%%
%									%
% 		Misc						%
%									%
%%%%%%%%%%%%%%%%%%%%%%%%%%%%%%%%%%%%%

\newcommand{\eqtext}[1]{\stackrel{\mathclap{\normalfont\mbox{#1}}}{=}} % write text over =
% could use a way to make the text smaller

\newcommand{\cat}[1]{\textbf{#1}}
\newcommand{\homset}[2]{\mathrm{Hom(#1,#2)}}

\DeclareMathOperator{\Hom}{Hom}
%\DeclareMathOperator[1]{\colim}{\underset{#1}{colim}}
\DeclareMathOperator{\colim}{colim}
%\newcommand[1]{\colim}{\underset{#1}\colim_op}
\DeclareMathOperator{\Fun}{Fun}

\renewcommand{\set}[1]{\{\,#1\,\}}

\newcommand{\fprod}[1]{\langle #1 \rangle}

\newcommand{\predsnop}[1]{\cat{Sets}^{\cat{#1}^{op}}}

\newcommand{\powerset}[1]{\mathcal{P}(#1)}

%\DeclareMathOperator{\eval}{eval}

% \newcommand{\coprod}[2]{\cat{#1} + \cat{#2}}


%%%%%%%%%%%%%%%%%%%%%%%%%%%%%%%%%%%%%%%%%%
% Two-way rule for adjunction
%TODO make this more usefull
%%%%%%%%%%%%%%%%%%%%%%%%%%%%%%%%%%%%%%%%%%
\ExplSyntaxOn
\NewDocumentEnvironment{adjunctions}{O{}}
 {
  \cs_set_eq:cN {@arraycr} \farin_arraycr:
  \keys_set:nn { farin/adjunction } { #1 }
  \begin{array}
   {
    @{ \hspace { \dim_eval:n { \l_farin_left_shift_dim + \l_farin_padding_dim } } }
    r
    @{ {\farin_strut:} \l_farin_symbol_tl {} }
    l
    @{ \hspace { \dim_eval:n { \l_farin_right_shift_dim + \l_farin_padding_dim } } }
   }
 }
 {
  \end{array}
 }
\keys_define:nn { farin/adjunction }
 {
  leftshift       .dim_set:N = \l_farin_left_shift_dim,
  leftshift       .initial:n = 0pt,
  rightshift      .dim_set:N = \l_farin_right_shift_dim,
  rightshift      .initial:n = 0pt,
  padding         .dim_set:N = \l_farin_padding_dim,
  padding         .initial:n = 6pt,
  symbol          .tl_set:N  = \l_farin_symbol_tl,
  symbol          .initial:n = \longrightarrow,
  verticalspacing .dim_set:N  = \l_farin_vertspac_dim,
  verticalspacing .initial:n = {3pt},
 }
\cs_new_protected:Npn \farin_strut:
 {
  \vrule height \dim_eval:n { \ht\strutbox + 1.2\l_farin_vertspac_dim }
         depth  \dim_eval:n { \dp\strutbox + \l_farin_vertspac_dim }
         width 0pt
 }
\makeatletter
\exp_args:NNo \cs_new:Npn \farin_arraycr:
 {
  \@arraycr\hline
 }
\makeatother
\ExplSyntaxOff


%%%%%%%%%%%%%%%%%%%%%%%%%%%%%%%%%%%%%%%%%%
%%%%%%%%%%%%%%%%%%%%%%%%%%%%%%%%%%%%%%%%%%

\begin{document}
\selectlanguage{slovene}
\frontmatter
\setcounter{page}{1} %
\renewcommand{\thepage}{}       % preprecimo težave s številkami strani v kazalu
\newcommand{\sn}[1]{"`#1"'}                    % dodal Solina (slovenski narekovaji)



%%%%%%%%%%%%%%%%%%%%%%%%%%%%%%%%%%%%%%%%
%naslovnica
 \thispagestyle{empty}%
   \begin{center}
    {\large\sc Univerza v Ljubljani\\%
      Fakulteta za računalništvo in informatiko}%
    \vskip 10em%
    {\autfont \tauthor\par}%
    {\titfont \ttitle \par}%
    {\vskip 3em \textsc{DIPLOMSKO DELO\\[5mm]
    INTERDISCIPLINARNI UNIVERZITETNI\\ ŠTUDIJSKI PROGRAM PRVE STOPNJE\\ RAČUNALNIŠTVO IN MATEMATIKA}\par}%

    \vfill\null%
    {\large \textsc{Mentor}: prof.\ dr.  Andrej Bauer\par}%
    {\vskip 2em \large Ljubljana, 2017 \par}%
\end{center}
% prazna stran
%\clearemptydoublepage      % dodal Solina (izjava o licencah itd. se izpiše na hrbtni strani naslovnice)

%%%%%%%%%%%%%%%%%%%%%%%%%%%%%%%%%%%%%%%%
%copyright stran
\thispagestyle{empty}
\vspace*{8cm}

\noindent
{\sc Copyright}. 
Rezultati diplomske naloge so intelektualna lastnina avtorja in Fakultete za računalništvo in informatiko Univerze v Ljubljani.
Za objavo in koriščenje rezultatov diplomske naloge je potrebno pisno privoljenje avtorja, Fakultete za računalništvo in informatiko ter mentorja.

\begin{center}
\mbox{}\vfill
\emph{Besedilo je oblikovano z urejevalnikom besedil \LaTeX.}
\end{center}
% prazna stran
\clearemptydoublepage


%%%%%%%%%%%%%%%%%%%%%%%%%%%%%%%%%%%%%%%%
% stran 3 med uvodnimi listi
\thispagestyle{empty}
\vspace*{4cm}

\noindent
Fakulteta za računalništvo in informatiko izdaja naslednjo nalogo:
\medskip
\begin{tabbing}
\hspace{32mm}\= \hspace{6cm} \= \kill


Tematika naloge:
\end{tabbing}
Besedilo teme diplomskega dela študent prepiše iz študijskega informacijskega sistema, kamor ga je vnesel mentor. V nekaj stavkih bo opisal, kaj pričakuje od kandidatovega diplomskega dela. Kaj so cilji, kakšne metode uporabiti, morda bo zapisal tudi ključno literaturo.
\vspace{15mm}

\vspace{2cm}

% prazna stran
\clearemptydoublepage

% zahvala
\thispagestyle{empty}\mbox{}\vfill\null\it%
\noindent
Na tem mestu zapišite, komu se zahvaljujete za izdelavo diplomske naloge. Pazite, da ne boste koga pozabili. Utegnil vam bo zameriti. Temu se da izogniti tako, da celotno zahvalo izpustite.
\rm\normalfont

% prazna stran
\clearemptydoublepage

Svojim staršem

%%%%%%%%%%%%%%%%%%%%%%%%%%%%%%%%%%%%%%%%
% posvetilo, če sama zahvala ne zadošča :-)
\thispagestyle{empty}\mbox{}{\vskip0.20\textheight}\mbox{}\hfill\begin{minipage}{0.55\textwidth}%

\normalfont\end{minipage}

% prazna stran
\clearemptydoublepage


%%%%%%%%%%%%%%%%%%%%%%%%%%%%%%%%%%%%%%%%
% kazalo
\pagestyle{empty}
\def\thepage{}% preprecimo tezave s stevilkami strani v kazalu
\tableofcontents{}

% prazna stran
\clearemptydoublepage


%%%%%%%%%%%%%%%%%%%%%%%%%%%%%%%%%%%%%%%%
% povzetek
\addcontentsline{toc}{chapter}{Povzetek}
\chapter*{Povzetek}

\noindent\textbf{Naslov:} \ttitle
\bigskip

\noindent\textbf{Avtor:} \tauthor
\bigskip

%\noindent\textbf{Povzetek:} 
\noindent V vzorcu je predstavljen postopek priprave diplomskega dela z uporabo okolja \LaTeX. Vaš povzetek mora sicer vsebovati približno 100 besed, ta tukaj je odločno prekratek.
Dober povzetek vključuje: (1) kratek opis obravnavanega problema, (2) kratek opis vašega pristopa za reševanje tega problema in (3) (najbolj uspešen) rezultat ali prispevek magistrske naloge.

\bigskip

\noindent\textbf{Ključne besede:} \tkeywords.
% prazna stran
\clearemptydoublepage

%%%%%%%%%%%%%%%%%%%%%%%%%%%%%%%%%%%%%%%%
% abstract
\selectlanguage{english}
\addcontentsline{toc}{chapter}{Abstract}
\chapter*{Abstract}

\noindent\textbf{Title:} \ttitleEn
\bigskip

\noindent\textbf{Author:} \tauthor
\bigskip

%\noindent\textbf{Abstract:} 
\noindent This sample document presents an approach to typesetting your BSc thesis using \LaTeX. 
A proper abstract should contain around 100 words which makes this one way too short.
\bigskip

\noindent\textbf{Keywords:} \tkeywordsEn.
\selectlanguage{slovene}
% prazna stran
\clearemptydoublepage

%%%%%%%%%%%%%%%%%%%%%%%%%%%%%%%%%%%%%%%%
\mainmatter
\setcounter{page}{1}
\pagestyle{fancy}

\chapter{Uvod}
Kaj je teorija kategorij? Kako se razlikuje od običajnega pogleda na matematiko? Običajno se v matematiki obravnava in preučuje matematične strukture in preslikave med njimi, ki ohranjajo strukturo. Na primer grupe, kolobarje, topološke prostore. Včasih opazimo, da se v bistvu vse lastnosti, ki jih naši objekti imajo in to kako se obnašajo, da izraziti s tem, kakšne so možne transformacije teh objektov. Teorija kategorij poskuša uporabiti in posplošiti to idejo na vse, kar se obnaša na tak način. Ne zanima nas namreč več zgradba teh struktur, ampak le tiste lastnosti, ki jih lahko razberemo iz transformacij med strukturami. Kaj se dogaja, na primer pri kompozitumu katerih od teh transformacij in kaj lahko povemo s tem. Na neki način teorija kategorij združuje različna področja v matematiki in poskuša nanje pogledati z enotno perspektivo. Ta posplošitev seveda ne more rešiti vseh specifičnih problemov nekega področja, nam pa da bolj globalni pogled na stvari in včasih tudi lahko prenese pristop k reševanju določenega problema na povsem drugo področje, če ga le znamo pogledati s pravilno perspektivo.

\section{Osnovne Definicije}

\begin{definicija}
\emph{Kategorija} sestoji iz:
\begin{itemize}
\item \emph{objektov} : $A,B,C,X,Y,\ldots$
\item \emph{morfizmov} : $f,g,h,\ldots$
\item za vsak morfizem imamo podana dva objekta: $$dom(f), \quad cod(f).$$
ki jima pravimo \emph{domena} in \emph{kodomena} morfizma $f$. Pišemo:
$$f\colon A \to B$$
kjer sta $A = dom(f)$ in $B = cod(f)$.
Pravimo da $f$ gre od $A$ do $B$.
\item Za vsaka morfizma $f \colon A \to B$ in $g \colon B \to C$, torej taka da velja $cod(f)$ = $dom(g)$, obstaja morfizem $g\circ f \to C$, ki mu pravimo \emph{kompozitum} $f$ in $g$.
%
\[
\begin{tikzcd}
A \arrow[r, "f"] \arrow[rd, "g \circ f"']  & B  \arrow[d, "g"] \\
				& C
\end{tikzcd}
\] 
%
\item za vsak objekt $A$ obstaja morfizem
$$1_A : A \to A$$
ki mu pravimo \emph{identiteta} na $A$.		
\end{itemize}
Za vse te podatke morata veljati naslednji dve pravili:
\begin{itemize}
\item asociativnost: Za vsake $f : A \to B, g : B \to C, h : C \to D$ velja
$$h \circ (g \circ f) = (h \circ g) \circ f$$
\item enota: Za vsak $f : A \to B$ velja
$$f \circ 1_A = f = 1_B \circ f$$
\end{itemize}
\end{definicija}
%
Zbirko objektov kategorije včasih označujemo z $Obj(\cat{C})$ in zbirko morfizmov z $Arr(\cat{C})$.

\section{Primeri kategorij}

\begin{primer}
Osnovni primer kategorije, na katerega se lahko vedno sklicujemo, je kategorija množic in funkcij med njimi. Označimo ga s $\cat{Set}$. Za kategorijo se vedno najprej vprašamo: kaj so objekti in kaj so morfizmi? Pri $\cat{Sets}$ so objekti množice in morfizmi funkcije.
Izpolnjena morata biti pogoja asociativnosti in enote.
Kompozitum morfizmov je kompozitum funkcij, ki je asociativen, kar vemo iz teorije množic.
Vlogo identitete igra identitetna funkcija, ki jo lahko vedno definiramo in zanjo velja $f \circ id_A = f = id_B \circ f$ za vsako funkcijo $f : A \to B$, kjer je $id_A : A \to A$ definirana kot $id_A(x) = x$ za vsak element $x \in A$.
\end{primer}

\begin{primer}
Še ena kategorija, ki jo že poznamo, je $\cat{Set}_{fin}$. To je kategorija končnih množic in funkcij med njimi. Zakaj je to res kategorija? Objekti so očitno končne množice, torej so morfizmi funkcije med končnimi množicami. Kompozitum takih funkcij je očitno tudi takšna funkcija. Identiteta je podedovana iz kategorije $\cat{Set}$ in bo v tem primeru funkcija na končni množici, torej res morfizem v tej kategoriji. Asociativnost kompozituma sledi iz asociativnosti kompozituma funkcij. To je tudi primer kategorije, kjer so objekti strukturirane množice ter morfizmi funkcije, ki to strukturo ohranjajo.
\end{primer}

\begin{primer}
Delno urejena množica je množica $P$, opremljena z relacijo, ki se jo ponavadi označuje z $\leq$ in je:
\begin{itemize}
\item refleksivna: $\forall x \in P \ x \leq x$
\item antisimetrična: $x \leq y \ \& \ y \leq x \Rightarrow x = y$
\item tranzitivna: $x \leq y \ \& \ y \leq z \Rightarrow x \leq z$
\end{itemize}
Morfizem delno urejenih množic $P$ in $Q$ je \emph{monotona} funkcija
$$m : P \to Q$$
kar pomeni, da za vse $x,y \in P$ iz $x \leq y$ sledi $m(x) \leq m(y)$. Ali je to kategorija? Naravni kandidat za identiteto je identitetna funkcija $1_P : P \to P$, ki je monotona, saj iz $x \leq y$ sledi $x \leq y$.
Kompozitum dveh monotonih funkcij $m : P \to Q$ in $n : P \to Q$ je monotona funkcija, saj za $x \leq y$ zaradi monotonosti $m$ velja $m(x) \leq m(y)$, zaradi monotonosti $n$ pa velja $n(m(x)) \leq n(m(y))$. Imamo torej kategorijo, ki jo označujemo s $\cat{Pos}$, delno urejenih množic in monotonih funkcij.
\end{primer}

\begin{primer}
Monoid $(M, \bullet)$ je množica opremljena z binarno operacijo množenja, za katero drži:
\begin{itemize}
\item zaprtost: $\forall x,y \in M : x \bullet y \in M$
\item asociativnost: $\forall x,y,z \in M : x \bullet ( y \bullet z ) = ( x \bullet y ) \bullet z$
\item obstoj enote: obstaja tak $e \in M$ tako, da $\forall x \in M : e \bullet x = x \bullet e = x$
\end{itemize}
Homomorfizmi monoidov so funkcije $f : M \to N$ za katere velja:
\begin{itemize}
\item $f(e_M) = e_N$
\item $f(x \bullet y) = f(x) \bullet f(y)$
\end{itemize}
V tej kategoriji bo vlogo identitete igral identitetni homomorfizem $1_M : M \to M$. Iz teorije monoidov je znano, da je kompozitum homomorfizmov tudi homomorfizem, torej imamo novo kategorijo monoidov in homorfizmov med njimi, ki jo označujemo z $\cat{Mon}$.
\end{primer}

\begin{primer}
Objekti kategorije niso nujno strukturirane množice in morfizmi niso nujno funkcije. Kategoriji kjer pa to drži, pravimo \emph{konkretna kategorija}. Vse kategorije, ki smo jih do zdaj obravnavali, so primeri konkretnih kategorij.
Poglejmo si še primer ne-konkretne kategorije. Naj bo $\cat{Rel}$ kategorija, kjer so objekti množice in morfizmi binarne relacije. Torej, morfizem $f : A \to B$ je podmnožica kartezičnega produkta $f \subseteq A \times B$. Identiteta je identitetna relacija na množici.
$$ 1_A = \set{(a,a) \in A \times A \mid a \in A} \subseteq A \times A$$
Za relaciji $R \subseteq A \times B$ in $S \subseteq B \times C$, definiramo njun kompozitum $S \circ R$ kot:
$$(a,c) \in S \circ R \quad \Leftrightarrow \quad \exists b \ (a,b) \in R \ \& \ (b,c) \in S.$$
%
Najprej mora veljati, da je tako definiran kompozitum res spet morfizem te vrste, kar jasno je, saj je množica elementov kartezičnega produkta, torej binarna relacija. Da je to res kategorija, je potrebno preveriti še, da je kompozitum identitete s poljubnim kompatibilnim morfizmom, res nazaj isti morfizem in, da je kompozitum morfizmov asociativen. Recimo torej, da imamo množico $A$ in morfizem $R \subseteq A \times B$. Če njun kompozitum zapišemo po definiciji, je:
$$R \circ 1_A = \set{(a,b) \in A \times B \mid \exists a \in A (a,a) \in A \ \& \ (a,b)\in R} = R.$$
Da bi preverili asociativnost, denimo, da imamo morfizme $Q:A \to B$, $R:B \to C$ in $S: C \to D$. Sedaj velja:
\begin{align*}
(a,d) \in S \circ (R \circ Q) &\Leftrightarrow \exists c \in C \ (a,c) \in R \circ Q \ \& \ (c,d) \in S \\
&\Leftrightarrow \exists c \in C \ \exists b \in B \ (a,b) \in Q \ \& \ (b,c) \in R \ \& \ (c,d) \in S \\
&\Leftrightarrow \exists b \in B \ (a,b) \in Q \ \& \ (b,d) \in S \circ R \\
&\Leftrightarrow (a,d) \in (S \circ R) \circ Q.
\end{align*}

\end{primer}

\begin{primer}
Kaj bi bil primer "`minimalistične"' kategorije? Kategorije z majhnim številom objektov ali morfizmov. Ker mora za vsak objekt obstajati identiteta, mora vsaka kategorija imeti najmanj toliko morfizmov, kolikor je objektov. Najmanjše število objektov, ki jih lahko imamo je 0. Ali je kategorija z 0 objekti in 0 morfizmi res kategorija? Za vsak objekt, ki jih ni, obstaja identiteta in za vsaka dva kompatibilna morfizma, ki ju ni, obstaja njun kompozitum. To torej je kategorija. 

Kaj pa kategorija z enim objektom? Imeti mora en objekt in najmanj en morfizem, namreč identiteto na tem objektu. To kategorijo, z enim samim morfizmom, pogosto označujemo z $\mathbf{1}$, ko je iz konteksta jasno, da gre za kategorijo (je ta del res potreben?). Kategorijo z dvema objektoma in enim neidentitetnim morfizmom med objektoma označujemo z $\mathbf{2}$. Ti dve kategoriji lahko predstavimo z diagramoma:

\begin{equation}
\begin{tikzcd}
\bullet \arrow[loop right] &&&  \arrow[loop left] \bullet \arrow[r] & \bullet \arrow[loop right]
\end{tikzcd}
\end{equation}
%
Identitetnih morfizmov se ponavadi ne riše. Kategorija $\cat{3}$ bi izgledala takole:
%
\begin{equation}
\begin{tikzcd}
\bullet \arrow[r] \arrow[rd] & \bullet \arrow[d] \\
& \bullet
\end{tikzcd}
\end{equation}
\end{primer}

\begin{primer}
Naj bo $(P, \leq)$ delno urejena množica. Pokažimo, da je tudi to kategorija. Najprej se moramo vprašati, kaj so objekti v tej kategoriji in kaj so morfizmi?
Imamo množico elementov $p,q \in P$, med katerimi lahko imamo relacijo $p \leq q$, ki je refleksivna, antisimetrična in tranzitivna. Dobimo idejo, da za objekte vzamemo elemente $P$ in podamo morfizem med $p$ in $q$ natanko takrat, ko v $P$ velja $p \leq q$.
Torej: 
\begin{itemize}
\item objekti: elementi množice $P$
\item morfizmi: morfizem $p \rightarrow q \Leftrightarrow p \leq q$
\end{itemize}
Potrebno je preveriti, če so izpolnjeni aksiomi za kategorijo:

\begin{itemize}
\item Za vsak objekt $p \in P$ potrebujemo morfizem $1_p : p \to p$. Ali obstaja tak morfizem? Obstaja, saj za vsak $p$ velja $p \leq p$, kar nam da želeno identiteto.
\item Za vsaka dva morfizma $p \to q$ in $q \to r$ mora obstajati kompozitum $p \to r$. Ali res obstaja? Obstaja, saj je relacija $\leq$ tranzitivna in iz $p \leq q$ in $q \leq r$ sledi $p \leq r$, kar nam da želeni kompozitum.
\end{itemize}
Vsaka delno urejena množica je torej svoja kategorija. Kategorije so v nekem smislu posplošene delno urejene množice.
\end{primer}



\begin{primer}
Vsako množico $A$ lahko obravnavamo kot kategorijo $\cat{Dis}(A)$, kjer za objekte vzamemo elemente $A$ in so edini morfizmi identitete na vsakem objektu. Taki kategoriji, kjer so edini morfizmi identitete pravimo \emph{diskretna} kategorija.
\end{primer}

\begin{primer}
Poglejmo še en primer, ki ga bralec mogoče že pričakuje. Objekti naj bodo topološki prostori in morfizmi naj bodo zvezne funkcije med njimi. Zveznim funkcijam, kot tudi včasih drugim funkcijam, ki ohranjajo struktuo, bomo rekli preslikave. Identiteta za poljuben topološki prostor $(X, \mathcal{T})$ je identitetna preslikava $1_X : (X, \mathcal{T}) \to (X, \mathcal{T})$, ki je zvezna. Osnovno dejstvo topologije je, da je kompozitum dveh zveznih funkcij spet zvezna funkcija. Torej topološki prostori res tvorijo kategorijo. Označujemo jo s $\cat{Top}$.
\end{primer}

\section{Različni tipi morfizmov}
Uvedemo prvo abstraktno definicijo v jeziku teorije kategorij, nečesa, kar je že poznano iz drugih področij matematike.

\begin{definicija} Naj bo \cat{C} poljubna kategorija. Morfizmu $f : A \to B$ pravimo \emph{izomorfizem}, če obstaja tak morfizem $g : B \to A$, da velja
$$g \circ f = 1_A \ in \ f \circ g = 1_B$$
Morfizmu $g$ pravimo \emph{inverz} morfizma f
\end{definicija}

\begin{trditev} Inverzi, ko obstajajo, so enolični.
\end{trditev}
\begin{dokaz}
Naj bo $f : A \to B$ izomorfizem in naj bosta $g,h: B \to A$ njegova inverza. Potem velja 
$$g = 1_A \circ g = h \circ f \circ g = h \circ 1_B = h.$$
\end{dokaz}

Ker so inverzi enolični, lahko inverz morfizma $f$ upravičeno označujemo z $f^{-1}$.

\begin{primer*}
Izomorfizmi v kategoriji $\cat{Sets}$ ustrezajo ravno bijektivnim preslikavam, saj kot vemo iz teorije množic, ima funkcija inverz, ravno kadar obstaja enoličen inverz te funkcije, ki se komponira v identiteto. 
\end{primer*}

\begin{primer*}
Vsak identitetni morfizem je izomorfizem, nima pa kategorija nujno drugih izomorfizmov kot identitetnih. Na primer, kategorija $\cat{2}$ ima samo en neidentitetni morfizem, ki pa nima inverza, torej ni izomorfizem.
\end{primer*}

\begin{primer}
Naj bo $(M,\cdot)$ monoid. Monoid si lahko interpretiramo kot kategorijo z enim samim objektom, kjer so morfizmi elementi monoida, ki imajo za domeno in kodomeno edini objekt te kategorije. Identiteta na tem objektu je enota monoida, kompozitum dveh morfizmov je produkt elementov, ki ju predstavljata. Torej, če sta $m,n \in M$, je njun kompozitum enak $m \cdot n \in M$. Asociativnost kompozituma sledi iz asociativnosti množenja v monoidu.

Lahko se vprašamo, ali imamo v tej kategoriji kake izomorfizme? Kaj bi to pomenilo? Radi bi dva taka morfizma $m,n$, da je njun kompozitum enak identiteti. Povedano drugače, radi bi dva elementa monoida, katerih produkt je enota. To pa je ravno definicija inverza. Torej, v monoidu (gledano kot kategorija) je morfizem izomorfizem natanko takrat, ko je obrnljiv element monoida. Ta razmislek nam pove tudi, da je grupa ravno kategorija z enim objektom, kjer je vsak morfizem izomorfizem.
\end{primer}

Primer s funkcijami nam naravno porodi novo vprašanje, saj kot vemo, je funkcija bijektivna ravno takrat, ko je surjektivna ter injektivna. Vprašamo se, kaj bi pa bili karakterizaciji teh dveh lastnosti v jeziku teorije kategorij? Izkaže se, da pridemo do malo bolj splošnih pojmov, ki jih predstavimo v naslednjih dveh definicijah.

\begin{definicija}
\emph{Epimorfizem} je tak morfizem $e : E \to A$, da za vsaka morfizma $f,g : A \to B$ iz $f \circ e = g \circ e$ sledi $f = g$.
\end{definicija}

\begin{definicija}
\emph{Monomorfizem} je tak morfizem $m : B \to M$, da za vsaka morfizma $f,g : A \to B$ iz $m \circ f = m \circ g$ sledi $f = g$.
\end{definicija}

Algebraično gledano, to da je morfizem $e$ epimorfizem, pomeni natanko to, da ga lahko pri kompoziciji krajšamo z desne: $fe = ge \implies f=g$. Obratno, če je morfizem $m$ monomorfizem, pomeni, da ga lahko krajšamo z leve: $mf = mg \implies f = g$.

\begin{primer}
Preverimo, da mono- in epi-morfizmi v $\cat{Sets}$ ustrezajo ravno injektivnim in surjektivnim funkcijam. Naj bo torej najprej $f : A \to B$ injektivna funkcija. Potem za vsaka $x,y \in A$ velja, da iz $f(x) = f(y)$ sledi $x = y$. Naj bosta sedaj $g,h : C \to A$ taki funkciji, da velja $f \circ g = f \circ h$. Torej za vsak $x \in C$ velja $f(g(x)) = f(h(x))$, iz česar iz injektivnosti $f$ sledi $g(x) = h(x)$, za vsak $x$, torej $g = h$. 
Privzemimo sedaj, da je $f$ monomorfizem. Naj bo $1 = \{\star\}$ in naj bosta $x,y : 1 \to A$. Funkcije iz množice $1$ v $A$ predstavljajo ravno elemente množice $A$. Ker pa velja $f \circ x = f \circ y \implies x = y$ velja tudi, da za vsaka $x,y \in A$ velja $f(x) = f(y) \implies x = y$ in je $f$ res injektivna. Naj bo sedaj $f : A \rightarrow B$ surjektivna funkcija. Naj bosta $g,h : B \to C$ taki funkciji, da velja $g \circ f = h \circ f$. Torej za vsak $y \in B$ velja $g(y) = h(y)$, saj zaradi surjektivnosti lahko najdemo $x \in A$, za katerega velja $y = f(x)$. Torej je $f$ res epimorfizem. Naj bo zdaj $f : A \to B$ epimorfizem in naj bosta $g,h : B \to 2$ definirani z naslednjima predpisoma
\[
g(x)=
\begin{cases}
1 \quad\text{;}\quad x \in \mathrm{Im}(f) \\
0 \quad\text{;}\quad x \notin \mathrm{Im}(f) \\
\end{cases}
\]
$$ h(x) = 1,$$
kjer je $\mathrm{Im}(f) := \set{y \in B \mid \exists x \in A \ y = f(x)}$.
Poglejmo si kompozituma $g \circ f$ in $h \circ f$. Velja 
$$(g \circ f)(x) = g(f(x)) = 1,$$
za vsak $x \in A$, saj je $f(x) \in \mathrm{Im}(f)$. Po drugi strani pa je tudi 
$$(h \circ f)(x) = h(f(x)) = 1.$$
Torej je za vsak $x \in A$, $(g \circ f)(x) = (h \circ f)(x)$, oziroma $g\circ f = h\circ f$. Ker pa je $f$ epimorfizem sledi $g = h$, kar pomeni, da velja $g(y) = h(y) = 1$ za vsak $y \in B$. Iz definicije $g$ sledi, da za vsak $y \in B$ velja $y \in \mathrm{Im}(f)$, torej je $f$ surjektivna.
\end{primer}

Iz tega primera bi lahko sklepali, da so epi- in mono-morfizmi vedno, v konkretni kategoriji, ravno surjektivne ter injektivne funkcije. Naslednji primer pokaže, da temu ni tako.

\begin{primer}
Naj bo $f : (\mathbb{N},+) \to (\mathbb{Z},+)$ homomorfizem monoidov, definiran s predpisom $f(n) = n$ za vsak $n \in \mathbb{N}$. Naj bosta $g,h: \mathbb{Z} \to M$ taka homomorfizma monoidov, da velja $g \circ f = h \circ f$. 
Velja torej
$$g(n) = g(f(n)) = h(f(n)) = h(n),$$
za $n \in \mathbb{N}$.
Ker sta $g,h$ homomorfizma monoidov velja
$$g(0) = h(0) = 0.$$
Zanimajo nas še vrednosti $g(-n)$ za $n \in \mathbb{N}$. 
Računamo:
\begin{align*}
g(0) = g(n - n) &= g(n) + g(-n) \\
\implies g(-n) &= -g(n).
\end{align*}
To pa pomeni
$$g(-n) = -g(n) = -h(n) = h(-n),$$
%
oziroma $g = h$, torej je $f$ epimorfizem.
\end{primer}

\begin{primer}
Ta razmislek nam da misliti, da je homomorfizem monoidov epimorfizem, če njegova slika generira celotno kodomeno. Generator monoida je taka podmnožica monoida, da lahko vsak element monoida generiramo s končnim številom operacij med elementi te podmnožice. Naj bosta $M,N$ monoida in $G \subseteq N$ naj bo generator monoida $N$. Denimo nato, da je $f: M \to N$ tak morfizem monoidov, da velja $G \subseteq \mathrm{Im}(f)$. Da pokažemo, da je $f$ epimorfizem, predpostavimo, da obstajata taka morfizma monoidov $g,h: N \to T$, da velja $g \circ f = h \circ f$. Vsak $y \in N$ lahko zapišemo kot 
$$y = a_1 a_2 \ldots a_n,$$
kjer so $a_1,a_2,\ldots,a_n \in G$ in $n \in \mathbb{N}$. Prav tako lahko  vsak $a_i$ zapišemo kot
$$a_i = f(x_i) \quad i=1,2,\ldots,n$$
za neke $x_i$ iz $M$. Zato velja:
\begin{align*}
g(y) &= g(a_1 a_2 \ldots a_n) \\
&= g(a_1) g(a_2) \ldots g(a_n) \\
&= g(f(x_1)) g(f(x_2)) \ldots g(f(x_n)) \\
&= h(f(x_1)) h(f(x_2)) \ldots h(f(x_n)) \\
&= h(a_1) h(a_2) \ldots h(a_n) \\
&= h(a_1 a_2 \ldots a_n) \\
&= h(y). \\
\end{align*}


\end{primer}


\section{Konstrukcije novih kategorij}

Sedaj, ko poznamo nekaj primerov, bi radi razširili svoj nabor kategorij. Kot znamo iz že znanih množic zgraditi nove, s pomočjo kartezičnega produkta, unije, preseka, tako želimo iz znanih kategorij zgraditi nove. Prvi tak konstrukt, ki v teoriji množic nima svojega točnega analoga, a se izkaže za izjemno pomembnega, je pojem dualne kategorije.


\begin{definicija}

\emph{Obratna} ali \emph{dualna} kategorija kategorije $\cat{C}$ se običajno označuje s $\cat{C}^{op}$ in je kategorija z isto zbirko objektov in zbirko morfizmov, kjer imajo vsi morfizmi zamenjano domeno in kodomeno. 
To pomeni, da imamo za morfizem $f : A \to B$ v $\cat{C}$ morfizem $f : B \to A$ v $\cat{C}^{op}$. Konceptualno je to kategorija, kjer so vsi morfizmi obrnjeni.
\end{definicija}

Prepričajmo se, da je to res kategorija. Identitete ostanejo iste, saj je domena enaka kodomeni. Kaj pa se zgodi s kompozitumi? Objekte in morfizme v dualni kategoriji se ponavadi označuje kar z enakimi oznakami, a da bodo stvari bolj jasne, uvedimo za trenutek naslednje oznake: za morfizem $f : A \to B$ v $\cat{C}$ pišimo $f^* : B^* \to A^*$ v $\cat{C}^{op}$. Tako dobimo zvezo med operacijami v $\cat{C}$ in $\cat{C}^{op}$.
\begin{align*}
(1_C)^* &= 1_{C^*} \\
(g \circ f)^* &= f^* \circ g^*
\end{align*}
Torej diagram v $\cat{C}$
\[ \begin{tikzcd}
A \arrow[r, "f"] \arrow[rd, "g \circ f"'] & B \arrow[d, "g"] \\
&	C \\
\end{tikzcd} \]
%
v $\cat{C}^{op}$ zgleda kot
%
\[ \begin{tikzcd}
A^* & \arrow[l, "f^*"'] B^* \\
& \arrow[lu, "f^* \circ g^*"] C^* \arrow[u, "g^*"']
\end{tikzcd} \]
%
Dualna kategorija nam predstavi pojem dualnosti, ki nam omogoča, da razne konstrukcije prenesemo v njihovo dualno obliko in tako iz ene konstrukcije dobimo dve.

\begin{primer*}
Tako dualnost smo že srečali, ko smo vpeljali pojma epimorfizma in monomorfizma. Naj bosta $e : E \to A$ epimorfizem in $m : D \to M$ monomorfizem, v kategoriji $\cat{C}$. Če postavimo njuna diagrama enega ob drugega:

\[ \begin{tikzcd}
E \ar[r, "e"] & A \ar[r, shift left, "f"] \ar[r, shift right, "f'"'] & B & C \ar[r, shift left, "g"] \ar[r, shift right, "g'"'] & D \ar[r, "m"] & M
\end{tikzcd} \]
postane jasno, da je prvi dualna verzija drugega, ali z drugimi besedami, epimorfizem je monomorfizem v dualni kategoriji.
\end{primer*}


\begin{primer}
\emph{Kategorija morfizmov} $\cat{C}^{\rightarrow}$ kategorije $\cat{C}$, je kategorija dobljena iz kategorije $\cat{C}$ tako, da za objekte vzamemo morfizme iz kategorije $\cat{C}$, na primer $f : A \to B$ in $f' : A' \to B'$ kar lahko predstavimo shematično:
\[ \begin{tikzcd}
A \arrow[d, "f"'] & A' \arrow[d, "f'"] \\
B & B'
\end{tikzcd} \]
Kako bi sedaj prešli iz $f$ v $g$? Ideja, ki se nam naravno porodi, je da povežemo obe stranici navideznega kvadrata z morfizmi v $\cat{C}$:
\[ \begin{tikzcd}
A \arrow[d, "f"'] \arrow[r, "g_1"] & A' \arrow[d, "f'"] \\
B \arrow[r, "g_2"'] & B'
\end{tikzcd} \]
Morfizem $f \to g$ je par morfizmov $(g_1, g_2)$ iz $\cat{C}$, tako da sta obe poti po kvadratu enaki, torej $g_2 \circ f = f' \circ g1$, čemur rečemo, da \emph{kvadrat komutira}.
Identiteta je par $(1_A, 1_B)$.
Da bo to kategorija, moramo biti sposobni definirati kompozitum takih morfizmov.
Recimo, da imamo naslednjo situacijo:
%
$$\begin{tikzcd}
A \arrow[d, "f"'] \arrow[r, "g_1"] & A' \arrow[d, "f'"] \arrow[r, "h_1"] & A'' \arrow[d, "f''"] \\
B \arrow[r, "g_2"'] & B' \arrow[r, "h_2"'] & B'' \\
\end{tikzcd}$$
%
kjer so $f, f', f''$ objekti v $\cat{C}^{\rightarrow}$, $(g_1, g_2), (h_1,h_2)$ pa morfizmi v $\cat{C}^{\rightarrow}$. Kaj bi bil kompozitum morfizmov $(h_1,h_2) \circ (g_1,g_2)$? Očitna izbira, ki se tudi izkaže za pravilno, je kompozitum po komponentah, oziroma $(h_1,h_2) \circ (g_1,g_2) = (h_1 \circ g_1, h_2 \circ g_2)$. Preveritmi moramo komutativnostni pogoj:
\begin{align*}
f'' \circ (h_1 \circ g_1) &= (f'' \circ h_1) \circ g_1 = \\
(h_2 \circ f') \circ g_1 &= h_2 \circ (f' \circ g1) = \\
h_2 \circ (g_2 \circ f) &= (h_2 \circ g_2) \circ f.
\end{align*}

\end{primer}

\begin{primer}
\emph{Rezine} $\cat{C}/B$ kategorije $\cat{C}$, nad objektom $B \in \cat{C}$. Ideja te kategorije je podobna ideji kategorije morfizmov, le da v tem primeru gledamo morfizme v $\cat{C}$, ki imajo kodomeno, torej kažejo v, objekt $B$. Na primer:
%
\[ \begin{tikzcd}
A \arrow[rd, "f"'] & & \arrow[ld, "f'"] A' \\
& B &
\end{tikzcd} \]
%
Ostale stvari delujejo podobno kot v kategoriji morfizmov. Morfizmi so ravno tako morfizmi v $\cat{C}$, a z eno zahtevo manj, saj sedaj ne bo potrebno poslati kodomene prvega morfizma v kodomeno drugega. V našem primeru bi bil morfizem iz $f$ v $f'$, morfizem $g : A \to A'$ v $\cat{C}$, tako da sledeči trikotnik komutira.
%
\[ \begin{tikzcd}
A \arrow[rd, "f"'] \ar[rr, "g"] & & \arrow[ld, "f'"] A' \\
& B &
\end{tikzcd} \]
oziroma z enačbo
$$f' \circ g = f.$$
Identiteta se podeduje iz $\cat{C}$, kompozitum pa deluje ravno tako, kot v kategoriji morfizmov. Če pogledamo malo bolj natančno, lahko vidimo, da ta konstrukcija izgleda kot neka "`podkonstrukcija"' kategorije morfizmov, če izmed vseh objektov kategorije $\cat{C}^{\rightarrow}$, vzamemo le tiste s kodomeno $B$. V tem primeru so morfizmi oblike $(g, 1_B)$ in vidimo, da komutativnostni kvadrati ustrezajo ravno komutativnostnim trikotnikom v $\cat{C}/C$.
%
\[ \begin{tikzcd}
A \ar[d, "f"'] \ar[r, "g"] & A' \ar[d, "f'"] \\
B \ar[r, "1_B"'] & B \\
\end{tikzcd} \]
%
Definiramo lahko tudi \emph{korezine} $B/\cat{C}$, kjer za objekte vzamemo morfizme v $\cat{C}$, ki kažejo iz $B$, oziroma tiste z domeno $B$. Ostale stvari potekajo podobno kot pri rezinah.
Poglejmo, kako lahko iz rezin dobimo korezine, kajti definiciji sta povezani prek dualnosti.
Diagrama v $\cat{C}/B$ ter $B/\cat{C}$ lahko predstavimo kot:
$$\begin{tikzcd}[column sep=small]
A \ar[dr, "f"'] \ar[rr, "\alpha"] & &  A' \ar[dl, "g"] & & A \ar[rr, "\alpha"] && A' \\
& B &  & & & B \ar[ul, "f"] \ar[ur, "g"'] & \\
\end{tikzcd}$$
Če pa pogledamo rezine v dualni kategorij $\cat{C}^{op}$, nad objektom $C$ dobimo diagram:
$$\begin{tikzcd}[column sep=small]
A && A' \ar[ll, "\alpha"'] \\
& B \ar[ul, "f"] \ar[ur, "g"'] &
\end{tikzcd}$$
v $\cat{C}$, kar pa predstavlja ravno korezine nad objektom $B$.

\end{primer}

\section{Funktorji}

V teoriji kategorij spoznamo abstraktne karakterizacije in konstrukcije, ki delujejo v določeni kategoriji. Radi bi konstrukcijo, ki smo jo spoznali v eni kategoriji, prenesli na druge kategorije. To bi storili v upanju, da je neke probleme lažje rešiti v drugi kategoriji in bomo znali rešitev prenesti nazaj v originalno kategorijio, kjer nas rešitev tega problema bolj zanima. V ta namen uvedemo definicijo, ki je v nekem smislu ključna pri uporabi teorije kategorij.

\begin{definicija}
Naj bosta $\cat{C}$ in $\cat{D}$ kategoriji. \emph{Funktor} $F : \cat{C} \to \cat{D}$ med kategorijama $\cat{C}$ in $\cat{D}$, je par preslikav
$$F_0 : \cat{C}_0 \to \cat{D}_0$$
med objekti kategorij in
$$F_1 : \cat{C}_1 \to \cat{D}_1$$
med morfizmi, tako da veljajo naslednje lastnosti:
\begin{itemize}
\item $F(f : A \to B) = F(f) : F(A) \to F(B)$
\item $F(1_A) = 1_{F(A)}$
\item za morfizma $f : A \to B, \ g : B \to C$ mora veljati:
$$F(g \circ f) = F(g) \circ F(f)$$
\end{itemize}
\end{definicija}

\begin{primer}
V vsaki kategoriji $\cat{C}$ imamo na voljo identitetni funktor, ki deluje na pričakovan način. Kompozitum funktorjev je spet funktor, saj za funktorja $F : \cat{C} \to \cat{D}$ in $G : \cat{D} \to \cat{E}$ in morfizem $f : A \to B$ velja:
$$G(F(f : A \to B)) = G(F(f) : F(A) \to F(B)) = G(F(f)) : G(F(A)) \to G(F(B))$$
ter
$$G(F(1_A)) = G(1_{F(A)}) = 1_{G(F(A))}$$
in še
$$G(F(g \circ f)) = G(F(g) \circ F(f)) = G(F(g)) \circ G(F(f)),$$
za $g : B \to C$. Imamo torej kategorijo, kjer so objekti kategorije in morfizmi med njimi funktorji. To kategorijo običajno označujemo s $\cat{Cat}$. Na ta način so funktorji posebni morfizmi med kategorijami.
\end{primer}

\begin{primer}
Za neko delno urejeno množico $P$, v kategoriji $\cat{Pos}$, lahko "`pozabimo"' strukturo urejenosti in vzamemo samo množico. Tej ideji pravimo \emph{pozabljivi funktor} $U : \cat{Pos} \to \cat{Set}$. Ta ideja je tudi bolj splošna, saj lahko za vsako konkretno kategorijo definiramo pozabljivi funktor tako, da vzamemo za vsak objekt samo množico, ki jo predstavlja.
\end{primer}

\begin{primer}
Primer pozabljivega funktorja, ki slika iz rezin v prvotno kategorijo, je funktor $U : \cat{C}/B \to \cat{C}$, ki tudi pozabi na gledani objekt $B$ tako, da si od vsakega objekta v $\cat{C}/B$, zapomni le njegovo domeno, morfizmi pa ostanejo kar enaki.
\end{primer}

\begin{primer}
Za vsak morfizem $g : B \to D$, lahko definiramo funktor $g_* : \cat{C}/B \to \cat{C}/D$, s predpisom $g_*(f) := g \circ f$,

$$\begin{tikzcd}
A \ar[d, "f"'] \ar[dr, "g \circ f"] & \\
B \ar[r, "g"'] & D \\
\end{tikzcd}$$

\end{primer}

\begin{primer}
Naj bosta $M$ in $N$ monoida in $f : M \to N$ homomorfizem monoidov. Vemo že, da je $f$ morfizem v kategoriji monoidov $\cat{Mon}$, a velja tudi, da je $f$ funktor med $M$ in $N$, če ju gledamo kot kategoriji. Ta funktor slika objekt prvega monoida v objekt drugega monoida in morfizme usklajeno s tem, kamor se slikajo elementi monoida. Ker gre za homomofrizem, slika enoto v enoto ter produkt v produkt, kar pomeni, da gre res za funktor. V tem smislu so funktorji neke vrste posplošeni homomorfizmi.
\end{primer}


\section{Začetni in končni objekti}
V kategoriji $\cat{Set}$ množic in funkcij med njimi poznamo posebne tipe množic, kot na primer prazno množico in enojec. Poglejmo si abstraktizacijo teh dveh posebnih primerov v jezik teorije kategorij. Definicijo podamo s tako imenovano \emph{univerzalno lastnostjo}, ki določi objekte do izomorfizma natančno, s pomočjo njihovega odnosa do ostalih objektov v kategoriji.

\begin{definicija}
V poljubni kategoriji $\cat{C}$ je objekt
\begin{itemize}
\item 0 \emph{začetni}, če za vsak objekt $C \in \cat{C}$ obstaja enoličen morfizem $$0 \to C.$$
\item 1 \emph{končni}, če za vsak objekt $C \in \cat{C}$ obstaja enoličen morfizem $$C \to 1.$$
\end{itemize}
\end{definicija}
Končni objekt je ravno začetni objekt v dualni kategoriji $\cat{C}^{op}$. Ker sta definiciji začetnega in končnega objekta podani z univerzalno lastnostjo, lahko pričakujemo, da bodo objekti podani do izomorfizma natančno. To pove naslednja trditev.

\begin{trditev}
Začetni in končni objekti so enolično določeni, do izomorfizma natančno.
\end{trditev}
\begin{dokaz}
Naj bosta $0$ in $\hat{0}$ začetna objekta v kategoriji $\cat{C}$. Potem obstajata enolična $g : 0 \to \hat{0}$ ter $\hat{g} : \hat{0} \to 0$. Ker sta $\hat{g} \circ g$ in identiteta na $0$ oba morfizma $0 \to 0$, obstaja pa natanko en morfizem $0 \to 0$, to pomeni, da morata biti enaka. Torej je $g$ izomorfizem.

Naj bosta zdaj $1$ in $\hat{1}$ končna objekta v $\cat{C}$. Potem obstajata enolična $g : 1 \to \hat{1}$ in $\hat{g} : \hat{1} \to 1$, kar pomeni, da je kompozitum $\hat{g} \circ g$ enoličen morfizem $1 \to \hat{1}$, oziroma identiteta. Torej je $g$ izomorfizem.
\end{dokaz}
Vidimo lahko, da sta dokaza za začetni in končni objekt potekala praktično enako. Gre seveda za delo dualnosti, ki jo bomo tudi formalno predstavili.

\begin{primer}
V kategoriji $\cat{Set}$ je začetni objekt prazna množica, saj za vsako množico $A$, obstaja natanko ena funkcija $! : \emptyset \to A$, ki nobenega elementa ne slika nikamor. 

Končni objekt v $\cat{Set}$ je enojec $1 = \set{*}$. Za vsako množico $A$ obstaja natanko ena funkcija $f : A \to 1$, ki je definirana s predpisom $f(x) = *$, za vse $x \in A$. Tu lahko vidimo, da je končni objekt določen "`le"' do izomorfizma natančno, saj je množica $\set{*}$ izomorfna vsakemu drugemu enojcu, s funkcijo 
$$f : \set{*} \to \set{a},$$
definirano s predpisom $f(*) = a$. Ta funkcija je očitno bijekcija, torej izomorfizem v kategoriji $\cat{Set}$.
\end{primer}

% Dodaj primere: Pos, Mon, Top.

\begin{primer}
Ali lahko najdemo končni objekt v kategoriji $\cat{Pos}$? Označimo ta hipotetični končni objekt z $\textbf{1}$. Zanj bi moralo veljati, da za vsako drugo delno urejeno množico $P$ obstaja enolična monotona funkcija $f : P \to \textbf{1}$. Če za $1$ vzamemo kar enojec $\set{*}$, lahko vidimo, da taka monotona funkcija $f$ res obstaja za vsak $P$ in ker je končni objekt določen do izomorfizma natančno, so vsi drugi končni objekti izomorfni temu enojcu, kar pa v $\cat{Pos}$ velja le za vse druge enojce.
\end{primer}

\section{Posplošeni elementi}
Za popolno razumevanje neke množice potrebujemo natanko poznati vse njene elemente. To nam pove vse, kar lahko vemo o tej množici. Elemente neke množice $A$ pa lahko identificiramo s funkcijami $1 \to A$, saj vsako funkcijo $f : 1 \to A$ identificiramo glede na to, kam pošlje edini element. Z drugimi besedami, množica $A$ je izomorfna množici vseh funkcij iz $1$ v $A$. To množico označujemo s $\Hom_{\cat{Sets}}(1,A)$ in ji pravimo hom-set (boljši izraz ?). Takšne množice bodo ključnega pomena pri Yonedovi lemi.
Pri ostalih matematičnih strukturah pa pogosto ni dovolj le poznavanje elementov te strukture. Na primer, za poznavanje topološkega prostora moramo poznati še okolice točk v tem prostoru in nam samo poznavanje točk ne pove ničesar o kvalitativnih lastnostih tega prostora. Kateri morfizmi pa so potrebni za poznavanje nekega objekta? To vprašanje nas privede do naslednje definicije.

\begin{definicija}
\emph{Posplošeni element} objekta $A \in \cat{C}$ je poljuben morfizem 
$$t : T \to A,$$
iz nekega \emph{testnega objekta} $T \in \cat{C}$.
\end{definicija}
Kot je bilo že omenjeno, nam posplošeni elementi razkrijejo dodatno strukturo, kar ponazori naslednji primer.

\begin{primer}
Recimo, da imamo dve delno urejeni množici $X$ in $A$, z naslednjo ureditvijo:
\begin{align*}
X = \set{x \leq y, x \leq z} \\
A = \set{a \leq b \leq c} \\
\end{align*}
Obe množici imata po 3 elemente, a vidimo, da nimata identične strukture. Torej, med njima imamo monotono bijektivno funkcijo $f : X \to A$, definirano s predpisom:
$$f(x) = a, \quad f(y) = b, \quad f(z) = c,$$
a ta funkcija \textbf{ni}(legitimno odebeljena pisava?) izomorfizem v $\cat{Pos}$. Ti dve strukturi v resnici nista izomorfni, a kako to pokazati? En način je s tako imenovanimi \emph{invariantami}. To so lastnosti neke strukture, ki se ohranjajo z izomorfizmi, oziroma jih imajo vse izomorfne strukture enake. Invariante se da definirati na lep način s posplošenimi elementi. V našem primeru vidimo, da je invarianta "`število elementov"', enaka za obe množici, kar ustreza temu, da morfizmi iz enojca \textbf{1} ne ločujejo med njima. Poglejmo si namesto tega "`\textbf{2}-elemente"' teh množic. To so morfizmi iz množice $\textbf{2} = \set{0 \leq 1}$ v naši množici. Takih morfizmov v množico $X$ je 5, in sicer trije morfizmi, ki slikajo oba elementa v isti element ter še dva dodatna morfizma
$$0 \mapsto x, 1 \mapsto y \qquad 0 \mapsto x, 1 \mapsto z.$$
Po drugi strani imamo za morfizme $\textbf{2} \to A$, poleg konstantnih morfizmov, še tri dodatne:
$$0 \mapsto a, 1 \mapsto b \qquad 0 \mapsto b, 1 \mapsto c \qquad 0 \mapsto a, 1 \mapsto c.$$
Tako je invarianta, ki jo lahko poimenujemo kar "`število morfizmov iz \textbf{2}"' za množico $X$ enaka 5, za množico $A$ pa 6, torej lahko sklepamo, da množici nista izomorfni v $\cat{Pos}$.
\end{primer}
%
Posplošeni elementi so uporabni tudi za "`testiranje"' določenih lastnosti. Poglejmo si na primer diagrame naslednje oblike:
%
$$\begin{tikzcd}[column sep=large]
X \ar[r, shift left, "x"] \ar[r, shift right, "x'"'] & A \ar[r, "f"] & B
\end{tikzcd}$$
Tukaj je $f$ monomorfizem, natanko takrat, ko za vsaka morfizma $x, x'$ iz $f \circ x = f \circ x'$ sledi $x = x'$, ali z drugimi besedami, $f$ je "`injektiven na posplošenih elementih"'.
%

Podobno, da lahko za diagram oblike:
$$\begin{tikzcd}[column sep=large, row sep=large]
A \ar[d, "g"'] \ar[r, "f"] & B \ar[d, "\alpha"] \\
D \ar[r, "\beta"'] & D
\end{tikzcd}$$
trdimo, da komutira, mora veljati, da je 
$$\alpha \circ f \circ  t = \beta \circ g \circ t$$
za vsak posplošen element $t : T \to A$, kajti potem to velja tudi za posplošen element $1_A : A \to A$. Posplošeni elementi so posebej koristni za testiranje takih in podobnih kategoričnih lastnosti.

\section{Produkti}
Kot smo to že storili, poskusimo znano konstrukcijo iz teorije množic, posplošiti na poljubno kategorijo. V $\cat{Set}$ lahko za poljubni množici $A$ in $B$ vzamemo njun kartezični produkt $A \times B$, ki je definiran kot množica vseh urejenih parov 
$$A \times B := \set{(a,b) \mid a \in A, b \in B}.$$
Definirani imamo tudi dve koordinatni projekciji $\pi_A : A\times B \to A$ in $\pi_B : A \times B \to B$, s predpisoma $\pi_A(a,b) = a$ in $\pi_B(a,b) = b$. Za vsak element $x \in A \times B$ velja $x = (\pi_A(x), \pi_B(x))$. Kot smo že videli, lahko elemente množice $A \times B$ identificiramo z morfizmi $1 \to A \times B$, kar predstavimo z diagramom:

$$ \begin{tikzcd}[row sep=huge, column sep=large]
&  1 \ar[dl, "a"'] \ar[d, dashed, "\text{(a,b)}", font=\large] \ar[dr, "b"] & \\
A & \ar[l, "\pi_A"] A \times B \ar[r, "\pi_B"'] & B \\
\end{tikzcd} $$

Če enojec $1$ zamenjamo s posplošenim elementom, dobimo naslednjo definicijo.

\begin{definicija}
\emph{Produkt objektov} $A,B \in \cat{C}$ je objekt $P$, skupaj z morfizmoma
$$\begin{tikzcd}
A & P \ar[l, "p_A"'] \ar[r, "p_A"] & B,
\end{tikzcd}$$
z naslednjo univerzalno lastnostjo: Za vsak objekt $X$ iz $\cat{C}$ in morfizma $f : X \to A$, $g : X \to B$ obstaja natanko en morfizem $u : X \to P$, tako da naslednji diagram:
%
$$\begin{tikzcd}
& X \ar[ld, "f"'] \ar[d, dashed, "u"] \ar[rd, "g"] & \\
A & P \ar[l, "p_A"] \ar[r, "p_B"'] & B
\end{tikzcd}$$
%
komutira. Zapisano z enačbami:
$$f = p_A \circ u \quad g = p_B \circ u.$$
Morfizmoma $p_A$ in $p_B$ pravimo \emph{(koordinatni) projekciji}. Morfizem $u$ ponavadi označujemo z $\langle f,g \rangle$.
\end{definicija}
%
Če v kategoriji obstaja produkt poljubnih dveh objektov, pravimo, da ta kategorija \emph{ima binarne produkte} (mogoče dvojiške).
%
Kot je običajno za definicije podane z univerzalno lastnostjo, bo veljala naslednja trditev.
%
\begin{trditev}
Produkti so enolični do izomorfizma natančno.
\end{trditev}
\begin{dokaz}
Naj bosta $P$ in $R$ oba produkta objektov $A$ in $B$, s produktnima projekcijama $p_A, p_B$ za $P$ ter $r_A, r_B$ za $R$. Ker je $P$ produkt $A$ in $B$, obstaja enoličen morfizem $u : R \to P$, da velja 
$$r_A = p_A \circ u, r_B = p_B \circ u.$$
Obratno, ker je $R$ produkt $A$ in $B$, obstaja enoličen $v : P \to R$, da veljata zvezi
$$p_A = r_Au, \quad p_B = r_Bu.$$
Stanje prikažemo v diagramu:
%
$$\begin{tikzcd}
& R \ar[dd, dashed, bend right, "u"'] \ar[dl, "r_A"'] \ar[dr, "r_B"] & \\
A & & B \\
& P \ar[uu, dashed, bend right, "v"'] \ar[ul, "p_A"] \ar[ur, "p_B"'] & \\
\end{tikzcd}$$
%
Morfizem $u \circ v $ je torej enolični morfizem iz $P$ v $P$. To pa pomeni, da mora biti identitetni morfizem. Enako velja za $v \circ u$, torej sta si inverza in sta $P$ in $R$ izomorfna.
\end{dokaz}

Zaradi zgornje trditve lahko produkt $A$ in $B$ upravičeno označujemo z $A \times B$.
%
\begin{primer}
Preverimo, da naš motivacijski zgled s kartezičnim produktom množic res ustreza univerzalni lastnosti produkta. Naj bo $A \times B$ kartezični produkt množic $A$ in $B$, opremljen s koordinatnima projekcijama $\pi_A$ in $\pi_B$, definiranima kot prej. Denimo, da obstaja množica $X$ s funkcijama $f : X \to A$, $g : X \to B$. Definirajmo funkcijo 
$$\langle f,g \rangle : X \to A \times B,$$
s predpisom $\langle f,g \rangle(x) := (f(x),g(x))$. Velja:
$$ (\pi_A \circ \langle f,g \rangle)(x) = \pi_A(f(x),g(x)) = f(x) $$
in 
$$ (\pi_B \circ \langle f,g \rangle)(x) = \pi_B(f(x),g(x)) = g(x) $$
Naj bo sedaj $h : X \to A \times B$ tak, da velja:
$$\pi_A \circ h = f, \ \pi_B \circ h = g.$$
Ker $h$ slika v kartezični produkt, ga lahko zapišemo kot:
$$h(x) = (h_1(x),h_2(x)),$$
kjer sta: $h_1 : X \to A$, $h_2 : X \to B$.
Iz te in zgornje enakost sledi:
$$h_1(x) = f(x)$$
in $$h_2(x) = g(x).$$
torej je res $h = \langle f,g \rangle$.
\end{primer}
%
Ravno tako, kot produkt dveh objektov, lahko definiramo produkt treh ali več objektov. Naj bo na primer $(C_i)_{i \in I}$ družina objektov, indeksirana po neki indeksni množici $I$ (lahko neskončni). Produkt družine $(C_i)_{i \in I}$ je objekt
$$\prod_{i \in I}C_i,$$
skupaj z družino morfizmov 
$$(p_j : \prod_{i \in I}C_i \to C_j)_{j \in I},$$
z univerzalno lastnostjo, da za vsak objekt $X$, z morfizmi $(x_i : X \to C_i)_{i \in I}$, obstaja natanko en morfizem $u : X \to \prod_{i \in I}C_i$, tako da velja
$$x_j = p_j \circ u,$$
za vsak $j \in I$. Oziroma da diagram:

$$\begin{tikzcd}
X \ar[d, dashed, "u"] \ar[dr, "x_j"] & \\
\prod_{i \in I}C_i \ar[r, "p_j"'] & C_j \\
\end{tikzcd}$$
komutira.
%
Definiramo lahko tudi eniški produkt. Eniški produkt objekta je objekt sam, brez dodatnih morfizmov. Ničelni produkt v kategoriji je končni objekt te kategorije, saj za vsak objekt obstaja natanko en morfizem v končni objekt, da nič dodatnega ne komutira. Če v kategoriji $\cat{C}$ obstaja produkt poljubne končne družine objektov pravimo, da $\cat{C}$ \emph{ima končne produkte}.
%
\begin{primer}
Naj bosta $\cat{C}$ in $\cat{D}$ kategoriji. \emph{Produkt kategorij} $\cat{C} \times \cat{D}$ je prav tako kategorija. Objekti so oblike $(C,D)$, kjer sta $C \in \cat{C}$ in $D \in \cat{D}$ in morfizmi oblike $(f,g) : (C,D) \to (C',D')$, kjer je $f : C \to C'$ morfizem v $\cat{C}$ ter $g : D \to D'$ morfizem v $\cat{D}$. Identitete ter kompozitume definiramo po komponentah, torej:
\begin{equation*}
1_{(C,D)} = (1_C,1_D),
\end{equation*}
\begin{equation*}
(f,g) \circ (f',g') = (f \circ f', g \circ g'),
\end{equation*}
za $C \in \cat{C}$, $D \in \cat{D}$ ter $f,f'$ morfizma v $\cat{C}$ in $g,g'$ morfizma v $\cat{D}$.
Za produkt $\cat{C} \times \cat{D}$ imamo dva \textit{projekcijska funktorja}:
\[
\begin{tikzcd}
\cat{C} & \arrow[l, "\pi_{\cat{C}}"'] \cat{C} \times \cat{D} \arrow[r, "\pi_{\cat{D}}"] & \cat{D},
\end{tikzcd}
\]
definirana na objektih kot $\pi_{\cat{C}}(C,D) = C$ in $\pi_{\cat{D}}(C,D) = D$, ter na morfizmih $\pi_{\cat{C}}(f,g) = f$ in $\pi_{\cat{D}}(f,g) = g$.
\end{primer}

\section{Dualnost}
Povejmo (za trenutek še brez dokaza) dve trditvi, ki nam opišeta in utemeljia pojem dualnosti v kategoriji.

\begin{trditev} \emph{(Formalna dualnost)}. 
Za vsak stavek $\Sigma$, v jeziku teorije kategorij, če $\Sigma$ sledi iz aksiomov kategorij, potem sledi tudi njegov dualni stavek $\Sigma^*$.
\end{trditev}

\begin{trditev} \emph{(Konceptualna dualnost)}.
Za vsako trditev $\Sigma$ o kategorijah, če $\Sigma$ drži za vse kategorije, potem drži tudi dualna trditev $\Sigma^*$.
\end{trditev}

Kar lahko potegnemo iz teh dveh trditev je, da za vsako trditev, ki jo dokažemo za poljubno kategorijo, "`zastonj"' dobimo še njeno dualno trditev, brez dodatnega dela. Tako bi lahko na primer pri dokazu, da so začetni in končni objekti določeni do izomorfizma natančno uporabili dejstvo, da je končni objekt dualni pojem začetnega objekta in naredili dokaz le za enega izmed njiju.
Ta ideja nam pove, da smo z obravnavanjem neke konstrukcije v teoriji kategorij (na primer univerzalne lastnosti produkta) hkrati obravnavali tudi njen dual. V takem primeru dualni konstrukciji dodamo predpono "`ko-"'. To nas pripelje do naslednjega poglavja (definicije?).

\subsection{Koprodukti}
\begin{definicija}
Naj bosta $A$ in $B$ objekta v kategorji $\cat{C}$. \emph{Koprodukt} $A$ in $B$ je objekt $Q$, skupaj z morfizmoma $q_A : A \to Q$, $q_B : B \to Q$, ki zadoščajo naslednji univerzalni lastnosti. Za vsak objekt $X$, skupaj z morfizmoma $x_A : A \to X$, $x_B : B \to X$, obstaja enoličen morfizem $u : Q \to X$, za katerega naslednji diagram:
%
$$\begin{tikzcd}
& X & \\
A \ar[ur, "x_A"] \ar[r, "q_A"'] & Q \ar[u, dashed, pos=0.35, "v"']  & B \ar[l, "q_B"] \ar[ul, "x_B"'] \\
\end{tikzcd}$$
komutira. V enačbah se to glasi:
$$x_A = v q_A \quad x_B = v q_B$$
\end{definicija}
Enako kot za produkte velja naslednja trditev.
\begin{trditev}
Koprodukti so enolično določeni do izomorfizma natančno.
\end{trditev}
\begin{dokaz}
Uporabimo dejstvo, da to velja za produkte in da je koprodukt dual produkta.
\end{dokaz}

Koprodukt $A$ in $B$ zato označujemo z $A + B$.
\begin{primer}
Ali lahko najdemo koprodukte v kategoriji $\cat{Set}$? 
Veljati mora univerzalna lastnost koprodukta.
Poskusimo z množico 
$$A+B := \set{(a,1) \mid a \in A} \cup \set{(b,2) \mid b \in B}$$
in funkcijama 
$$i_A : A \to A + B, \quad i_A(a) = (a,1)$$ 
in 
$$i_B : B \to A+B, \quad i_B(b) = (b,2).$$
Tej množici pravimo \emph{disjunktna unija} $A$ in $B$. Velja še, da lahko vsak element te množice zapišemo kot $(x,k)$, kjer je $x \in A \cup B$ in $k \in \set{1,2}$. Recimo torej, da imamo množico $X$ in funkciji $f : A \to X$, $g: B \to X$. Definirajmo $u: A+B \to X$ kot:
\[
u((x,k)) = 
	\begin{cases}
		f(x) &\quad ;k = 1 \\
		g(x) &\quad ;k = 2 \\
	\end{cases}
\]
Sedaj očitno velja $f(a) = u(i_A(a))$ in $g(b) = u(i_B(b))$. Da preverimo enoličnost, recimo da obstaja še neka druga funkcija $h : A+B \to X$, da velja $f = h \circ i_A$, $g = h \circ i_B$. Brez izgube splošnosti predpostavimo, da je $x \in A$. Tedaj velja
$$h((x,k)) = h(i_A(x)) = f(x) = u(i_A(x)) = u((x,k)),$$
torej $u = h$.
\end{primer}

\begin{primer}
Tako kot za produkte lahko za poljubni kategorji $\cat{C}$, $\cat{D}$ definiramo njun koprodukt $\cat{C} + \cat{D}$, z inkluzijskima funktorjema $\iota_1 : \cat{C} \to \cat{C} + \cat{D}$, $\iota_2 : \cat{C} \to \cat{C} + \cat{D}$
\end{primer}

\section{Hom-sets}
Ustavimo se na kratko pri tej zelo pomembni temi, ki smo jo že srečali in bo postala ključnega pomena kasneje.


Spregovorimo najprej nekaj besed o velikostih kategorij. Kot je bilo razvidno iz jezika, uporabljenega pri definiranju kategorije, ko smo govorili o zbirki objektov in zbirki morfizmov, lahko njena kardinalnost preseže mejo tega, kar lahko opišemo z množicami. Res, če pogledamo kategorijo $\cat{Set}$ in se spomnemo na Russellov paradoks, postane jasno, da množice ne bodo dovolj. V primerih, ko pa je mogoče definirati množico objektov in množico morfizmov, pravimo, da gre za \emph{majhno kategorijo}, v nasprotnem primeru je kategorija \emph{velika}. Primeri majhnih kategorij so recimo končne kategorije ali diskretne kategorije. Problem velikih kategorij je, da je oteženo definiranje preslikav med njimi in v njih, saj nimamo na voljo vseh orodij iz teorije množic. Pomemben razred kategorij, kjer pa mnogo teh orodij je na voljo, je sledeč.

\begin{definicija}

Naj bo $\cat{C}$ taka kategorija, da je za vsaka objekta $A,B \in \cat{C}$ morfizmov med njima toliko, da jih lahko spravimo v množico. Taki kategoriji pravimo \emph{lokalno majhna kategorija}. To množico označujemo z:
$$\Hom_{\cat{C}}(A,B) := \set{f \in Arr(\cat{C}) \mid dom(f) = A, cod(f) = B}.$$
\end{definicija}
Če en objekt $A \in \cat{C}$ fiksiramo, dobimo funktor
$$\Hom(A,-) : \cat{C} \to \cat{Sets},$$
ki se ga imenuje \emph{(kovariantni) predstavljivi funktor}. Njegovo delovanje na objektih in morfizmih je definirano kot:
$$\Hom(A,-)(B) = \Hom(A,B)$$
ter
\begin{align*}
(A,-)(f : B \to C) = \Hom(A,f): \Hom(A,B) &\to \Hom(A,C) \\
g &\mapsto f \circ g, \\
\end{align*}

za diagram:
$$ \begin{tikzcd}
A \ar[r, "g"] \ar[rd, "f \circ g"'] & B \ar[d, "f"] \\
& C
\end{tikzcd} $$
v $\cat{C}$.
To je res funktor, saj je $\Hom(A,1_B)(f:A \to B) = 1_B \circ f = f$. Za diagram: 
$$ \begin{tikzcd}
& A \ar[dl, "f"'] & \\
B \ar[r, "g"'] & C \ar[r, "h"'] & D \\
\end{tikzcd} $$
pa velja:
$$
\Hom(A,h \circ g)(f) = h \circ g \circ f = \Hom(A,h)(g \circ f) = \Hom(A,h) \circ \Hom(A,g)(f).
$$
Na enak način lahko definiramo funktor
$$\Hom(-,A) : \cat{C}^{op} \to \cat{Set},$$
ki mu pravimo \emph{(kontravariantni) predstavljivi funktor},
kjer pa gledamo vse morfizme v objekt $A$. Funktorji tega tipa (iz dualne kategorije v \cat{Set}), nas bodo kasneje še posebej zanimali.

\section{Zožki in kozožki}

\subsection{Zožki}
Ideja zožkov je sledeča, zamislimo si, da imamo dve funkciji

$$\begin{tikzcd}
A \ar[r, shift left, "f"] \ar[r, shift right, "g"'] & B 
\end{tikzcd}$$

Radi bi zožali domeno $A$ na takšno podmnožico, da se $f$ in $g$ na njej ujemata. Označimo to množico z $E = \set{x \in A \mid f(x) = g(x)} \subseteq A$ in vložitev $E$ v $A$ z $e$, ki je torej definirana s predpisom $e(x) = x$ za vse $x \in E$. To, da je $x \in A$ v $E$, lahko ekvivalentno povemo tako, da obstaja morfizem $\hat{x} : 1 \to A$,(ta vejica res tukaj?) za katerega je $f \circ \hat{x} = g \circ \hat{x}$. To pa prav tako pomeni, da obstaja $\hat{x}' : 1 \to E$, da je $\hat{x} = e \circ \hat{x}' = \hat{x}'$, kajti taki so ravno vsi elementi iz $E$. Situacijo lahko ponazorimo z naslednjim diagramom:
$$\begin{tikzcd}
E \ar[r, "e"] & A \ar[r, shift left, "f"] \ar[r, shift right, "g"'] & B \\
1 \ar[u, dashed, "\hat{x}'"] \ar[ur, "\hat{x}"'] \\
\end{tikzcd}$$
Če uporabimo isto idejo kot prej in množico $1$ zamenjamo s posplošenim elementom, dobimo naslednjo definicijo.

\begin{definicija}
\emph{Zožek} dveh morfizmov $f, g : A \to B$ je par objekta $E$ in morfizma $e : E \to A$, da velja $f \circ e = g \circ e$, z univerzalno lastnostjo, da za vsak objekt $X$ in morfizem $x : X \to A$ za katerega velja $f \circ x = g \circ x$, obstaja enoličen morfizem $x' : X \to E$, da lahko $x$ razdelimo na $e$ in $x'$, oziroma $x = e \circ x'$.
Slika, ki jo lahko imamo v mislih je:
$$\begin{tikzcd}
E \ar[r, "e"] & A \ar[r, shift left, "f"] \ar[r, shift right, "g"'] & B \\
X \ar[u, dashed, "x'"] \ar[ur, "x"']
\end{tikzcd}$$

\end{definicija}

\begin{primer}
Preverimo, da naš motivacijski zgled ustreza definiciji zožka. Naj bosta $E$ in $e$ definirana kot zgoraj in naj bosta $X$ in $h:X \to A$ taka, da velja $f \circ h = g \circ h$, kar pomeni, da je $f \circ h(x) = g \circ  h(x)$ za vse $x \in X$, kar je seveda ekvivalentno temu, da je $h(x) \in E$. To pa pomeni, da bo funkcija $h' : X \to E$, definirana kot $h'(x) = h(x)$, razdelila $h$. Lahko se je prepričati, da je to edina funkcija, ki ustreza temu pogoju (bolj podrobno to dokazat?).
\end{primer}

\begin{trditev}
Če je $e : E \to A$ del zožka, je $e$ monomorfizem.
\end{trditev}
\begin{dokaz}
Naj bosta $x,y : X \to E$ takšna, da velja $e \circ x = e \circ y$. Označimo ta morfizem z $z := e \circ x = e \circ y$. Potem po definiciji $e$ velja $f \circ e \circ x = g \circ e \circ x$. Sledi, da obstaja enoličen $\hat{z} : X \to E$, da je $z = e \circ \hat{z} = e \circ x = e \circ y$. To pa pomeni, da je $x = y$.

$$\begin{tikzcd}
E \ar[r, "e"] & A \ar[r, shift left, "f"] \ar[r, shift right, "g"'] & B \\
X \ar[u, bend left, shift left, dashed, "\hat{z}"] \ar[u, shift left, "x"] \ar[u, shift right, "y"'] \ar[ur, "z"'] \\
\end{tikzcd}$$

\end{dokaz}


\subsection{Kozožki}

Kozožki so dualni koncept zožkov, zato lahko kar napišemo definicijo

\begin{definicija}
\emph{Kozožek} morfizmov $f,g:A \to B$ je objekt $Q$ in morfizem $q: B \to Q$ za katega je $q \circ f = q \circ g$ z naslednjo univezalno lastnostjo: Za vsak objekt $Z$ in morfizem $z : B \to Z$ za katerega velja $z \circ f = z \circ g$, obstaja enoličen morfizem $z' : Q \to Z$, da velja $z = z' \circ q$.

$$\begin{tikzcd}
A \ar[r, shift left, "f"] \ar[r, shift right, "g"'] & B \ar[r, "q"] \ar[dr, "z"'] & Q \ar[d, dashed, "z'"] \\
& & Z \\
\end{tikzcd}$$

\end{definicija}

Naslednja trditev sledi iz dualnosti.

\begin{trditev}
Če je $q : B \to Q$ del kozožka, je $q$ epimorfizem.
\end{trditev}

\begin{primer}
Kozožki so posplošitev pojma kvocientne množice, definirane z ekvivalenčno relacijo. Naj bo torej $A$ poljubna množica ter $R$ ekvivalenčna relacija na $A$, kar pomeni $R \subseteq A \times A$ z naslednjimi lastnostmi:
\begin{itemize}
\item refleksivnost: $\forall x \in A \ xRx$
\item simetričnost: $\forall x,y \in A \ xRy \implies yRx$
\item tranzitivnost: $\forall x,y,z \in A \ xRy \enspace\&\enspace yRz \implies xRz$
\end{itemize}
Imamo dve koordinatni projekciji $r_1 : R \to A$, $r_2 : R \to A$, definirani kot $r_1(x,y) = x$ in $r_2(x,y) = y$.
%
$$\begin{tikzcd}
& R \ar[dl, "r_1"'] \ar[d, hook] \ar[dr, "r_2"] & \\
A & A \times A \ar[l, "p_1"] \ar[r, "p_2"'] & A \\
\end{tikzcd}$$
Potem je kvocientna projekcija $\pi : A \to A/R$ kozožek $r_1$ in $r_2$.
Za vsak $(x,y) \in R$ mora veljati:
$$(\pi \circ r_1)(x,y) = (\pi \circ r_2)(x,y) \Leftrightarrow \pi(x) = \pi(y) \Leftrightarrow xRy \Leftrightarrow (x,y) \in R.$$ 
Denimo nato, da je za poljubno množico $X$, $f: A \to X$ tak, da je $f \circ r_1 = f \circ r_2$. To pomeni, da $f$ slika ekvivalentna elementa iz $A$ v isti element, ker pa $\pi$ ravno tako slika ekvivalentna elementa v isti element, lahko $\hat{f}:A/R \to X$ definiramo kot:
$$\hat{f}(\pi(x)) ) = f(x) = f(y) = \hat{f}(\pi(y)).$$
Torej je $\hat{f}$ dobro definirana. Ali je funkcija $\hat{f}$ edina taka? Recimo, da obstaja še neka druga $g : A/R \to X$, da zanjo velja $f = g \circ \pi$. Potem je za $xRy$, $(g \circ \pi)(x) = f(x) = f(y) = (g \circ\pi)(y)$. Ker pa je $\hat{f}(\pi(x)) = f(x) = g\pi(x)$, je $g = \hat{f}$.
%
$$\begin{tikzcd}
R \ar[r, shift left, "r_1"] \ar[r, shift right, "r_2"'] & A \ar[r, "\pi"] \ar[dr, "f"'] & A/R \ar[d, dashed, "\hat{f}"] \\
& & X \\
\end{tikzcd}$$

\end{primer}

\begin{primer}
V topologiji se pogosto pojavi situacija, kjer bi radi identificirali nekatere točke med sabo, kajti med njimi ne želimo razlikovati. To dosežemo tako, da generiramo ekvivalenčno relacijo, kjer točke med katerimi ne želimo razlikovati, označimo za ekvivalentne. Nov prostor zanimanja je tedaj kvocientni prostor, glede na to ekvivalenčno relacijo.  Naj bo torej $X$ neki topološki prostor in $R$ ekvivalenčna relacija na $X$. Definiramo funkcijo:
$$ q : X \to X/R,$$
ki jo imenujemo \emph{kvocientna projekcija} in slika točko $x \in X$ v njen ekvivalenčni razred. Od topologije $X/R$ zahtevamo vsaj to, da je kvocientna projekcija zvezna. Sledi, da smejo biti med odprtimi množicami v $X/R$ le take množice $V \subseteq X/R$, da je $q^{-1}(V)$ odprta v $X$. Za definicijo kvocientne topologije vzamemo kar vse take množice. Torej velja:
$$V \ \text{je odprta v} \ X/R \quad \overset{\text{def}}{\Leftrightarrow} \quad q^{-1}(V) \ \text{je odprta v} \ X.$$
Trdimo, da je $q$ kozožek koordinatnih projekcij $r_1,r_2$, ekvivalenčne relacije $R$.
Naj bo torej $f : X \to Y$ zvezna funkcija. Potem za vsako odprto množico $U \subseteq Y$ velja: $f^{-1}(U)$ je odprta v $X$. Iščemo tako zvezno funkcijo $\overline{f} : X/R \to Y$, da bo veljalo $f(x) = \overline{f}(q(x))$ za vsak $x \in X$.
$$\begin{tikzcd}
R \ar[r, shift left, "r_1"] \ar[r, shift right, "r_2"'] & X \ar[r, "q"] \ar[dr, "f"'] & X/R \ar[d, dashed, "\overline{f}"] \\
& & Y \\
\end{tikzcd}$$
Da bo $\overline{f}$ zvezna, mora za vsako odprto množico $U \subseteq Y$ veljati, da je $\overline{f}^{-1}(U)$ odprta v $X/R$, iz česar sledi, da mora biti $q^{-1}(\overline{f}^{-1}(U))$ odprta v $X$. Recimo, da obstaja še neka $h : X/R \to Y$, za katero velja $f(x) = h(q(x))$. Elementi $X/R$ so ravno ekvivalenčni razredi točk v $X$, oziroma $q(x)$ za nek $x \in X$. Ker velja $\overline{f}(q(x)) = h(q(x))$ za vse $x$, velja torej za vse elemente iz $X/R$, kar pa pomeni, da je $h = \overline{f}$. Funkcija $q$ je torej res kozožek $r_1$ in $r_2$.

\end{primer}

O kozožku si lahko mislimo, da zoži $B$ z identifikacijo vseh parov $f(a) = f(b)$. To naredi na "`najboljši"' način, tako, da vsako drugo zožanje $f$ in $g$ lahko speljemo skozi $Q$ (mogoče preveč po domače napisano?).


\section{Povleki in potiski}
Radi bi nadaljevali serijo posploševanja konceptov iz teorije množic pri čemer smo soočeni z naslednjo situacijo. Podani imamo dve funkciji z isto kodomeno, a morebiti različno domeno. Na primer $f:A \to C$ in $g:B \to C$. Sedaj bi želeli obravnavati le tiste elemente $(a,b) \in A \times B$, za katere velja $f(a) = g(b)$. V istem duhu kot smo to do sedaj že večkrat storili, elemente množic identificiramo s funkcijami iz enojca, nato pa te funkcije zamenjamo s posplošenimi elementi, kar nas pripelje do naslednje definicije.

\subsection{Povlek}

\begin{definicija}
Naj bosta $f : A \to C$ in $g : B \to C$ morfizma v kategoriji $\cat{C}$. \emph{Povlek} $f$ in $g$ je objekt $P$, skupaj z morfizmoma $p_1 : P \to A$ in $p_2 : P \to B$, takšnima, da tako imenovani "`pullback kvadrat"' (ustrezna terminologija?) komutira. Torej $f \circ p_1 = g \circ p_2$. Pri tem je $P$ izpolnjuje univerzalno lastnost, da za vsake $X$, $x_1 : X \to A$, $x_2 : X \to B$, za katere velja $f \circ x_1 = g \circ x_2$, obstaja enoličen morfizem $u : X \to P$, tako da velja $x_1 = p_1 \circ u$ in $x_2 = p_2 \circ u$. To shematično prikažemo z diagramom:

$$\begin{tikzcd}
X
\arrow[drr, bend left, "x_2"]
\arrow[ddr, bend right, "x_1"']
\arrow[dr, dotted, "u" description] & & \\
& P \arrow[r, "p_2"] \arrow[d, "p_1"']
& B \arrow[d, "g"] \\
& A \arrow[r, "f"']
& C
\end{tikzcd}$$
\end{definicija}

Na prvi pogled povlek deluje podobno kot produkt $A$ in $B$, le da smo omejeni še z morfizmoma $f$ in $g$. Ta podobnost ni slučajna.

\begin{primer}
Poglejmo si, kako bi glede na zgornjo razpravo lahko konstruirali povlek v kategoriji $\cat{Set}$. Denimo, da imamo funkciji $f : A \to C$ in $g : B \to C$. Potrebujemo množico $P$ skupaj s funkcijama $p_1 : P \to A$, $p_2 : P \to B$, ki bi ustrezale definiciji. Za prvi približek vzemimo množico $A \times B$, skupaj s koordinatnima projekcijama in poglejmo, kako bi jo morali popraviti, da bi izpolnjevala pogoj $f \circ p_1 = g \circ r_2$. Veljati bi moralo :
$$(f \circ p_1)(x,y) = (g \circ p_2)(x,y) \Leftrightarrow f(x) = g(y).$$
Definiramo :
$$P := \set{(x,y) \in A \times B \mid f(x) = g(y)}$$
s podedovanima projekcijama, ki ju tudi poimenujemo $p_1$ in $p_2$ in preverimo, če res izpolnjujejo univerzalno lastnost. Recimo, da obstajata $z_1 : Z \to A$ in $z_2 : Z \to B$ taki, da velja $f \circ z_1 = g \circ z_2$. Definirajmo $u : Z \to P$ kot $u(z) = (z_1(z), z_2(z))$
Situacija je sledeča:
%
$$\begin{tikzcd}
Z
\arrow[drr, bend left, "z_2"]
\ar[dr, "u"]
\arrow[ddr, bend right, "z_1"'] & & \\
& P \arrow[r, "p_2"] \arrow[d, "p_1"']
& B \arrow[d, "g"] \\
& A \arrow[r, "f"']
& C
\end{tikzcd}$$
%
Potem velja 
$$(p_1 \circ u)(z) = p_1(z_1(z),z_2(z)) = z_1(z)$$ 
in 
$$(p_2 \circ u)(z) = p_2(z_1(z),z_2(z)) = z_2(z),$$ 
za vsak $z \in Z$. Denimo sedaj, da obstaja še neka druga funkcija $v : Z \to P$, za katero je $z_1 = p_1 \circ v$, $z_2 = p_2 \circ v$. Poglejmo, kaj mora veljati za funkcijo $v$. Naj bo $v(z) = (a,b) \in P$ za neki $z \in Z$. Veljati mora $z_1(z) = (p_1 \circ v)(z) = p_1(x,y) = x$. Po drugi strani pa je $z_2(z) = (p_2 \circ v)(z) = p_2(x,y) = y$. Torej je $v(z) = (x,y) = (z_1(z),z_2(z))$, oziroma $v = u$.
\end{primer}

Povlek je torej v $\cat{Set}$ skrčitev kartezičnega produkta na tiste pare, ki se ujemajo glede na neki dve funkciji.

\begin{trditev}
Naj bo $\cat{C}$ kategorija s končnimi produkti in zožki, potem $\cat{C}$ ima povleke.
\end{trditev}
\begin{dokaz}
Naj bosta $f : A \to C$, $g : B \to C$ morfizma v kategoriji $\cat{C}$. Kot smo to storili z množicami, vzemimo za prvi približek produkt objektov $A$ in $B$ in premislimo, kako bi ga morali popraviti, da pridemo do povleka. Produkt pride opremljen s projekcijama, ki ju označimo z $\pi_1 : A \times B \to A$ in $\pi_2 : A \times B \to B$. Sedaj vzemimo zožek morfizmov $f \circ \pi_1$ in $g \circ \pi_2$, ki ga poimenujmo $e : E \to A \times B$. Situacija je sledeča:

$$\begin{tikzcd}
E
\ar[dr, "e"] \\
& A \times B \arrow[r, "\pi_2"] \arrow[d, "\pi_1"']
& B \arrow[d, "g"] \\
& A \arrow[r, "f"']
& C
\end{tikzcd}$$
oziroma v skrčeni (pullback) obliki:

$$\begin{tikzcd}
& E \arrow[r, "e\pi_2"] \arrow[d, "e\pi_1"']
& B \arrow[d, "g"] \\
& A \arrow[r, "f"']
& C
\end{tikzcd}$$
Recimo sedaj, da obstaja $X$ opremljen z morfizmoma $x_1 : X \to A$, $x_2 : X \to B$ tako, da velja $f \circ x_1 = g \circ x_2$. Par morfizmov $x_1,x_2$ lahko identificiramo z morfizmom
$$\langle x_1,x_2 \rangle : X \to A \times B,$$
za katerega velja
$$f \circ \pi_1 \circ \fprod{x_1,x_2} = g \circ \pi_2 \circ \fprod{x_1,x_2}.$$
Ker pa je $e$ zožek $f \circ \pi_1$ in $g \circ \pi_2$, obstaja enoličen morfizem $u : X \to E$, da velja 
$$\fprod{x_1,x_2} = e \circ u,$$
kar prikažemo shematično:
%
$$\begin{tikzcd}
X \ar[ddr, bend right, "x_1"'] \ar[drr, bend left, "x_2"] \ar[dr, dashed, "u"] \\
& E \arrow[r, "e\pi_2"] \arrow[d, "e\pi_1"']
& B \arrow[d, "g"] \\
& A \arrow[r, "f"']
& C
\end{tikzcd}$$
%
Napisano drugače:
$$x_1 = \pi_1 \circ e \circ u, \quad x_2 = \pi_2 \circ e \circ u$$
Kar pa pomeni, da je $E$, skupaj z morfizmoma $e \circ \pi_1$, $e \circ \pi_2$ ravno povlek $f$ in $g$.
\end{dokaz}

Objekt iz povleka se zaradi te povezave s produktom včasih označuje z $A \times_C B$.
Iz trditve vidimo, da je v povleku zakodirana vsa informacija, ki jo imata produkt ter zožek. Kot bomo videli, so vsi trije konstrukti primeri splošnejšega pojma, ki mu pravimo limita.



\subsection{Potiski}

Definicja potiska je seveda dualna definiciji povleka in jo lahko kar napišemo.

\begin{definicija}
Naj bosta $f: C \to A$ in $g : C \to B$ morfizma v kategoriji $\cat{C}$. \emph{Potisk} $f$ in $g$ je objekt $Q$, skupaj z morfizmoma $q_1 : A \to Q$, $q_2 : B \to Q$, tako da diagram:
$$\begin{tikzcd}
C \ar[r, "g"] \ar[d, "f"'] & B \ar[d, "q_2"] \\
A \ar[r, "q_1"'] & Q \\
\end{tikzcd}$$
komutira. Ta konstrukcija ima univerzalno lastnost, da za vsak objekt $Z$ in morfizmoma $z_1 : A \to Z$, $z_2 : B \to Z$ za katera velja $z_1 \circ f = z_2 \circ g$, obstaja enolično določen morfizem $v : Q \to Z$, da naslednji diagram:

$$\begin{tikzcd}
C \ar[r, "g"] \ar[d, "f"'] & B \ar[d, "q_2"] \ar[ddr, bend left, "z_2"] & \\
A \ar[r, "q_1"'] \ar[drr, bend right, "z_1"'] & Q \ar[dr, dashed, "v"] & \\
& & Z 
\end{tikzcd}$$
komutira. Zapisano z enačbami:
$$z_1 = v \circ  q_1, \quad z_2 = v \circ q_2.$$
\end{definicija}

Naravna ideja, kako se potisk udejanja v kategoriji $\cat{Set}$ nas napelje na to, da je tako povezan s koprodukti in kozožki, kot je povlek povezan s produkti in zožki.

\begin{primer}
Recimo, da imamo dve funkciji v $\cat{Set}$:
$$ \begin{tikzcd}
A \ar[d, "f"'] \ar[r, "g"] & C \\
B & \\
\end{tikzcd} $$

Naj bo $B + C$ koprodukt $B$ in $C$. Sedaj identificiramo tiste elemente $b \in B$ in $c \in C$ za katere obstaja tak $a \in A$, da velja:
$$f(a) = b \quad \text{in} \quad g(a) = c.$$
S pogojem $f(a) \sim g(a)$ generiramo ekvivalnečno relacijo $\sim$ na $B + C$. Če glede na to ekvivalenčno relacijo definiramo kvocient $B + C/{\sim}$, dobimo potisk $f$ in $g$, ki se ga včasih označuje z $B +_A C$. Ta konstrukcija je v veliki meri dualna tisti za povleke v $\cat{Set}$, kar pa niti ni presenetljivo.

\end{primer}

\section{Limite, kolimite in eksponenti}


\subsection{Limite}

Konstrukcije, ki smo jih videli do sedaj, imajo vse med sabo nekaj skupnega. Pri vseh igra osrednjo vlogo objekt, opremljen z morfizmi, ki izpolnjujejo določene komutativnostne pogoje in je "`najboljši"' tak objekt, ki ustreza tem pogojem. Do sedaj smo to konfiguracijo objekta in morfizmov poimenovali diagram. Definirajmo pojem diagrama bolj točno.

\begin{definicija}
\emph{Diagram oblike} $\cat{J}$ v kategoriji $\cat{C}$ je funktor $D : \cat{J} \to \cat{C}$, kjer $\cat{J}$ imenujemo \emph{indeksna kategorija}. Objekte v indeksni kategoriji običajno označujemo z malimi tiskanimi črkami $i,j, ipd.$, vrednosti funktorja $D$ v teh objektih pa z $D_i,D_j, ipd.$


\emph{Stožec nad diagramom} $D$ je objekt $C$ iz $\cat{C}$, skupaj z družino morfizmov $(c_i : C \to D_i)$ iz $\cat{C}$ za vsak objekt $i \in \cat{J}$ tako, da za vsak morfizem $\alpha : i \to j$ iz $\cat{J}$ naslednji diagram:
%
$$\begin{tikzcd}[column sep=small]
& C \ar[dl, "c_i"'] \ar[dr, "c_j"] & \\
D_i \ar[rr, "D_\alpha"'] & & D_j \\
\end{tikzcd}$$
%
komutira. Napisano z enačbami $$c_j = D_\alpha \circ c_i.$$
%
Morfizem stožcev $\vartheta : (C, c_i) \to (C', c_i')$ je morfizem $\vartheta : C \to C'$ v $\cat{C}$ tak, da za vsak $i \in \cat{J}$ naslednji diagram:
%
$$\begin{tikzcd}
C \ar[dr, "c_i"'] \ar[r, "\vartheta"] & C' \ar[d, "c_i'"] \\
& D_i
\end{tikzcd}$$
%
komutira. Z enačbo se to glasi
$$c_i = c_i' \circ \vartheta.$$
%
Tako dobimo novo kategorijo $\cat{Cone}(D)$ stožcev nad $D$.
\end{definicija}
%
Diagrame $D$ si lahko predstavljamo kot "`slike oblike $\cat{J}$"' v kategoriji $\cat{C}$. Poglejmo si primer limite na neki enostavni kategoriji. Za indeksno kategorijo $\cat{J}$ vzamemo diskretno kategorijo z dvema objektoma $\cat{J} = \set{1,2}$. Stožec nad diagramom $D : \cat{J} \to \cat{C}$ sestoji iz objekta $C$ in dveh morfizmov $c_1 : C \to D_1$, $c_2 : C \to D_2$:
%
$$\begin{tikzcd}[column sep=normal, row sep=small]
& C \ar[dl, "c_1"'] \ar[dr, "c_2"] & \\
D_1 && D_2 \\
\end{tikzcd}$$
%
Morfizem stožcev $\vartheta : (C,c_i) \to (C', c_i')$ zgleda kot:
%
$$\begin{tikzcd}
& C' \ar[dl, "c_1'"'] \ar[d, "\vartheta"] \ar[dr, "c_2'"] & \\
D_1 & C \ar[l, "c_1"] \ar[r, "c_2"'] & D_2, \\
\end{tikzcd}$$
%
tako da trikotnika komutirata, oziroma z enačbami:
$$c_1' = c_1 \circ \vartheta, \quad c_2' = c_2 \circ \vartheta.$$
%
\begin{definicija}
\emph{Limita} nad diagramom $D : \cat{J} \to \cat{C}$ je končni objekt v kategoriji stožcev nad $D$. Limito diagrama označujemo kot
$$\lim_{i \in \cat{J}}D_i,$$
(mogoče rajši samo $\lim D$)
skupaj z morfizmi $p_i : \lim D \to D_i$.
\end{definicija}
%
Če v celoti razpišemo univerzalno lastnost limite nad $D$ dobimo, da za vsak drug stožec $(C,c_i)$ v $\cat{Cone}(D)$, obstaja enoličen morfizem $u : C \to \lim D$ tako, da za vse $i \in \cat{J}$ velja $c_i = p_i \circ u$. To lahko vizualno predstavimo kot:
%
$$\begin{tikzcd}
C \ar[rr, dashed, "u"] \ar[drrr, pos=7/11, "c_j"'] \ar[dr, "c_i"'] & & \lim D \ar[dl, crossing over, pos=1/3, "p_i"'] \ar[dr, "p_j"] \\
& D_i \ar[rr, "D_\alpha"'] & & D_j \\
\end{tikzcd}$$
%
Limito si lahko torej predstavljamo kot "`najbližji"' stožec nad diagramom $D$, kajti vsi drugi stožci morajo "`iti skozi"' limito. 

\begin{primer}
Nadaljujmo s primerom za diagrame iz diskretne kategorije na dveh objektih, ki jo poimenujmo kot zgoraj z $\cat{J}$. Kaj je limita nad $D : \cat{J} \to \cat{C}$? To je tak objekt $\lim D$ z morfizmi $p_1 : \lim D \to D_1$, $p_2 : \lim D \to D_2$, da za vsak objekt $X$, opremljen z morfizmoma $x_1 : X \to D_1$, $x_2 : X \to D_2$, obstaja enoličen morfizem $u : X \to \lim D$, da naslednja trikotnika:
%
$$\begin{tikzcd}
& X \ar[dl, "x_1"'] \ar[d, dashed, "u"] \ar[dr, "x_2"] & \\
D_1 & \lim D \ar[l, "p_1"] \ar[r, "p_2"'] & D_2 \\
\end{tikzcd}$$
komutirata. V tem diagramu pa lahko zagledamo ravno univerzalno lastnost produkta objektov $D_1$ in $D_2$. Torej kategorija $\cat{C}$ ima limite tipa $\cat{J}$ natanko takrat, ko ima binarne produkte.

\end{primer}

\begin{primer}
Poskusimo še z enostavnejšo indeksno kategorijo. Naj bo $\cat{J}$ prazna kategorija, brez objektov in brez morfizmov. Potem obstaja natanko en diagram $D : \cat{J} \to \cat{C}$, ki nobenega objekta ne pošlje nikamor. Limita $\lim D$ nad tem diagramom je objekt brez dodatnih morfizmov tako, da za vsak drug objekt $C \in \cat{C}$ obstaja natanko en morfizem $u : C \to \lim D$ in nič dodatnega ne komutira. Ta limita je torej ravno \emph{končni objekt} v $\cat{C}$.
\end{primer}
Iz tega primera je razvidno tudi, da limita diagrama ne obstaja nujno v neki kategoriji, saj nima vsaka kategorija končnih objektov. Če za neko indeksno kategorijo obstaja limita za vsak diagram $D : \cat{J} \to \cat{C}$, pravimo, da $\cat{C}$ \emph{ima limite tipa} $\cat{J}$.

\begin{primer}
Naj bo $\cat{J}$ enaka kategoriji $\cat{1} = \set{\ast}$ z enim objektom in enim morfizmom. Stožec nad diagramom $D : \cat{1} \to \cat{C}$ je objekt $C$, skupaj z morfizmom $c : C \to D$ (abuse notacije, dovolj razumljivo ?). Limita nad D je objekt $\lim D$ z morfizmom $p : \lim D \to D$ tako, da za vsak drug stožec $(C, c)$ obstaja enoličen morfizem $u : C \to \lim D$, da trikotnik:

$$\begin{tikzcd}[column sep=small]
C \ar[dr, "c"'] \ar[rr, dashed, "u"] & & \lim D \ar[dl, "p"] \\
& D & \\
\end{tikzcd}$$
komutira. V tej situaciji lahko razpoznamo rezine $\cat{C}/D$ nad objektom $D = D(\ast)$. Limita $\lim D$ je končni objekt v tej kategoriji.
\end{primer}

Kot pri vseh primerih limit, ki smo jih spoznali do tukaj, velja trditev o enoličnosti, ki nam pove, da ko smo enkrat našli limito, smo že našli "`ta pravo"', saj je vsaka druga izomorfna tej.
\begin{trditev}
Limite so enolične do izomorfizma natančno.
\end{trditev}
\begin{dokaz}
Naj bosta $L$ in $I$ limiti za diagram $D : \cat{J} \to \cat{C}$ za neko indeksno kategorijo $\cat{J}$. Ker je $L$ limita na $D$, je tudi stožec nad $D$, torej obstaja enoličen morfizem $v : I \to L$. Obratno, ker je $I$ stožec nad $D$, obstaja enoličen morfizem $u : L \to I$. To pa pomeni, da je $v \circ u$ enoličen morfizem $L \to L$, kar pomeni, da sta $u$ in $v$ izomorfizma in sta $L$ in $I$ izomorfna objekta.
\end{dokaz}

\begin{definicija}
Za kategorijo $\cat{C}$ pravimo, da je \emph{kompletna}, če ima vse majhne limite, kar pomeni, da za vsako majhno indeksno kategorijo $\cat{J}$ in diagram $D : \cat{J} \to \cat{C}$, obstaja limita $\lim D_j$ v $\cat{C}$.
\end{definicija}



\subsection{Kolimite}

Kot smo tega že vajeni, so limite dualni pojem limit. In tako kot limite posplošijo produkte, zožke, povleke, itd., ravno tako kolimite posplošijo koprodukte, kozožke, potiske, itd.

\begin{definicija} (Direktna definicija)

\emph{Kolimita} nad diagramom $D : \cat{J} \to \cat{C}$ je začetni objekt v kategoriji kostožcev nad $D$. Kolimito označujemo z
$$\colim_{j \in J} D_j.$$
\end{definicija}

Primeri kolimit so torej pričakovani in dualni primerom limit. Na primer, za diskretno indeksno kategorijo z dvema objektoma je kolimita nad diagramom iz te kategorije enaka koproduktu slik objektov. Kolimita iz indeksne kategorije brez objektov in brez morfizmov je začetni objekt.


Velja seveda naslednja trditev, ki sledi iz dualnosti.

\begin{trditev}
Kolimita je enolična, do izomorfizma natančno.
\end{trditev}

\begin{definicija}
Dualno kot za limite, za kolimite poznamo pojem \emph{kokompletnosti}, kar pomeni, da ima za poljubno majhno indeksno kategorijo $\cat{J}$, vsak diagram $D : \cat{J} \to \cat{C}$, kolimito $\colim D_j$ v $\cat{C}$.
\end{definicija}

\subsection{Eksponenti}

Ideja eksponentov je posplošitev funkcijskih prostorov v naslednjem smislu. Denimo, da imamo funkcijo:
$$f(x,y) : A \times B \to C,$$
kjer eksplicitno označimo spremenljivke $x$ iz $A$ in spremenljivke $y$ iz $B$. Če sedaj fiksiramo nek $a \in A$, dobimo funkcijo:
$$f(a,y): B \to C,$$
ki je element
$$f(a,y) \in C^B$$
vseh funkcij iz $B$ v $C$. Če parameter $a$ variiramo, dobimo funkcijo
$$\widetilde{f} : A \to C^B,$$
definirano s predpisom $a \mapsto f(a,y)$. Funkcija $\widetilde{f}$ je enolično določena z enačbo
$$\widetilde{f}(a)(b) = f(a,b).$$
V bistvu je vsaka funkcija 
$$\phi : A \to C^B$$
oblike $\phi = \widetilde{f}$ za nek $f : A \times B \to C$, saj ga lahko definiramo kot
$$f(a,b) := \phi(a)(b).$$
To pa pomeni, da imamo izomorfizem $\Hom$-množic:
$$\Hom_{\cat{Set}}(A \times B, C) \cong \Hom_{\cat{Set}}(A, C^B),$$
oziroma bijektivno korespondenco med funkcijami oblike $f : A\times B \to C$ in $\widetilde{f} : A \to C^B$. Če želimo posplošiti to idejo na poljubno kategorijo, bo potrebno to bijekcijo narediti eksplicitno. To storimo s pomočjo posebne funkcije imenovane \emph{evaluacija}, ki jo označimo z:
$$\mathrm{eval} : C^B \times C \to B$$
in definiramo s predpisom 
$$\mathrm{eval}(g,b) := g(b).$$
Za to funkcijo velja, da za vsako množico $A$ in funkcijo $f : A \times B \to C$, obstaja enolična funkcija
$$\widetilde{f} : A \to C^B,$$
tako da velja $\mathrm{eval} \circ (\widetilde{f} \times 1_B) = f$. To lahko z diagramom prikažemo kot:
$$\begin{tikzcd}
C^B &  C^B \times B \ar[r, "\mathrm{eval}"] & C \\
A \ar[u, "\widetilde{f}"] &  A \times B \ar[u, "\widetilde{f} \times 1_B"] \ar[ur, "f"'] & \\
\end{tikzcd}$$
Če sedaj izluščimo lastnosti množice $C^B$ in evaluacijske funkcije in to posplošimo na poljubno kategorijo, smo privedeni do naslednje definicije.

\begin{definicija}
Naj kategorija $\cat{C}$ ima binarne produkte. \emph{Eksponent} objektov B in C sestoji iz objekta 
$$C^B$$
in morfizma
$$\epsilon : C^B \times B \to C$$ imenovanega \emph{evaluacija}, da za vsak objekt $A$ in morfizem 
$$f : A \times B \to C,$$
obstaja enoličen morfizem 
$$\widetilde{f} : A \to C^B,$$
tako da velja
$$\epsilon \circ (\widetilde{f} \times 1_B) = f.$$
To prikažemo v podobnem diagramu kot v diskusiji zgoraj: (ali je potrebno res podvajati ta diagram)

$$\begin{tikzcd}
C^B &  C^B \times B \ar[r, "\epsilon"] & C \\
A \ar[u, dashed, "\widetilde{f}"] &  A \times B \ar[u, "\widetilde{f} \times 1_B"] \ar[ur, "f"'] & \\
\end{tikzcd}$$

Tu morfizmu $\widetilde{f}$ pravimo \emph{transponiranka} od $f$.
\end{definicija}

Za vsak morfizem $$g : A \to B^C$$ pišemo
$$ \overline{g} := \epsilon \circ (g \times 1_B) : A \times B \to C$$
in morfizem $\overline{g}$ tudi imenujemo \emph{transponiranka} od $g$. Po enoličnosti iz definicije potem velja
$$\widetilde{\overline{g}} = g$$
Velja pa tudi
$$\overline{\widetilde{f}} = f$$
za vsak $f : A \times B \to C$, kar pomeni, da je operacija transponiranja
$$(f : A \times B \to C) \mapsto (\widetilde{f} : A \to C^B)$$
inverz inducirani operaciji
$$(g : A \to C^B) \mapsto (\overline{g} = \epsilon \circ (g \times 1_B) : A \times B \to C),$$
kar nam da želeni izomorfizem
$$\Hom_{\cat{C}}(A \times B, C) \cong \Hom_{\cat{C}}(A, C^B).$$

\begin{definicija}
Kategorija se imenuje \emph{kartezično zaprta}, če ima vse končne produkte in eksponente.
\end{definicija}

\section{Funktorji in naravne transformacije}

\subsection{Morfizmi med kategorijami}

Neuradni moto teorije kategorij bi se lahko glasil: \emph{morfizmi so bistveni},
kar bi pomenilo, da nas ponavadi ne zanima toliko, kaj točno so objekti v neki specifični kategoriji, temveč kaj se dogaja z morfizmi med njimi.

Funktorji, kot vemo, so morfizmi v kategoriji $\cat{Cat}$ in kot taki, so lahko na primer monomorfizmi ali epimorfizmi. Ker lahko na monomorfizme gledamo kot na posplošene podmnožice (to je treba nekje bolj razložiti), pravimo monomorfizmu v $\cat{Cat}$ \emph{podkategorija}. Za funktorje pa poznamo tudi druge klasifikacije, ki so pogosto uporabne.

\begin{definicija}
Za funktor $F : \cat{C} \to \cat{D}$ pravimo, da je 
\begin{itemize}
\item \emph{injektiven na objektih}, če je preslikava objektov $F_0 : \cat{C}_0 \to \cat{D}_0$ injektivna, oziroma, da je \emph{surjektiven na objektih}, če je $F_0$ surjektivna.
\item \emph{injektiven}/\emph{surjektiven na morfizmih}, če je $F_1$ injektivna/surjektivna.
\item \emph{poln}, če je za vsaka objekta $A,B \in \cat{C}$
$$F_{A,B} : \Hom_{\cat{C}}(A,B) \to Hom_{\cat{D}}(F(A),F(B))$$
surjektivna.
\item \emph{zvest}, če je za vsaka $A,B \in \cat{C}$
$$F_{A,B} : \Hom_{\cat{C}}(A,B) \to Hom_{\cat{D}}(F(A),F(B))$$
injektivna.
\end{itemize}
\end{definicija}

\begin{primer}
Poglejmo si, zakaj tej pojmi niso med seboj ekvivalentni. Naj bo $\cat{C}$ kategorija in naj bo 
$$\nabla : \cat{C} + \cat{C} \to \cat{C}$$ 
kodiagonalni funktor, torej tak, ki obe "`kopiji"' pošlje v "`original"'.

$$\begin{tikzcd}[column sep=normal]
C \ar[r, "\iota_1"] \ar[dr, "1_C"'] & C + C \ar[d, "\nabla"] & C \ar[l, "\iota_2"'] \ar[dl, "1_C"] \\
& C & \\
\end{tikzcd}$$

Pokažimo, da je ta funktor zvest, ni pa injektiven na morfizmih. 
Za vsak morfizem $f : A \to B$ iz $\cat{C}$ obstajata v $\cat{C} + \cat{C}$ dva distinktna morfizma $\iota_1(f)$ in $\iota_2(f)$, ki ju $\nabla$ slika v isti morfizem, torej očitno ni injektiven na morfizmih. Obratno, za vsaka objekta $A,B$ v $\cat{C} + \cat{C}$ med njima obstaja morfizem, le če sta objekta v isti kopiji kategorije $\cat{C}$ in tak morfizem že obstaja v $\cat{C}$. V tem primeru ga $\nabla$ slika v "`isti"' morfizem.

\end{primer}

\subsection{Morfizmi med funktorji}
Sedaj, ko smo videli nekaj primerov in dobili malo občutka za funktorje in posploševanje v teoriji kategorij, je vprašanje, ki se morda naravno pojavi, ali lahko definiramo transformacije med funktorji? Izkaže se, da se to da storiti, če na funktorje gledamo kot na morfizme v posebni kategoriji funktorjev, potem morfizme med njimi imenujemo naravne transformacije(\emph ?). Predstavljamo si jih lahko kot različne načine s katerimi med seboj primerjamo funktorje. Ta smer razmišljanja nas pripelje do naslednje definicije.

\begin{definicija}
Naj bosta $\cat{C}$ in $\cat{D}$ poljubni kategoriji in naj bosta $F,G : \cat{C} \to \cat{D}$ funktorja med tema kategorijama.
\emph{Naravna transformacija} 
$$\vartheta : F \to G$$
iz $F$ v $G$, je družina morfizmov
$$(\vartheta_C : FC \to GC)_{C \in \cat{C}},$$
tako, da za vsak morfizem $f : C \to D$ naslednji diagram:

\begin{equation}
\begin{tikzcd}[sep=huge]
FC \arrow[r, "\vartheta_C"] \arrow[d, "F(f)"'] & GC \arrow[d, "G(f)"] \\
FD \arrow[r, "\vartheta_D"'] & GD
\end{tikzcd}
\end{equation}
komutira.
\end{definicija}

Pogosta situacija, kjer srečamo naravne transformacije, je sledeča. Recimo, da imamo v neki kategoriji dve različni "`konstrukciji"', ki sta med seboj povezani na način, ki je neodvisen od izbire objektov in morfizmov, povezava je namreč med "`konstrukcijama"'. Ti "`konstrukciji"' sta seveda funktorja in povezava med njima je naravna transformacija. Poglejmo si to na konkretnem primeru.

\begin{primer}
Denimo, da ima $\cat{C}$ produkte. Za neke objekte $A,B,C \in \cat{C}$ si poglejmo produkta
$$(A \times B) \times C \quad \text{in} \quad A \times (B \times C).$$
Ne glede na izbiro objektov $A,B,C$ obstaja izomorfizem
$$h : (A \times B) \times C \xrightarrow{\sim} A \times (B \times C).$$
Kaj pa pomeni, da je ta izomorfizem neodvisen od izbranih objektov? Za neki morfizem $f : A \to A'$ dobimo komutativen kvadrat:
%
$$ \begin{tikzcd}
(A \times B) \times C \ar[r, "h_A"] \ar[d] & A \times (B \times C) \ar[d] \\
(A' \times B) \times C \ar[r, "h_A'"'] & A' \times (B \times C) 
\end{tikzcd} $$
Torej v resnici imamo izomorfizem med konstrukcijama :
$$(- \times B) \times C \cong - \times (B \times C).$$
To pa sta v resnici samo funktorja $\cat{C} \to \cat{C}$ in naš izomorfizem je v resnici naravna transformacija med tema funktorjema. Seveda lahko ta dva funktorja razširimo do funktorjev v vseh treh argumentih
$$(- \times -) \times - : \cat{C}^3 \to \cat{C}$$
in
$$ - \times (- \times -) : \cat{C}^3 \to \cat{C},$$
med katerima tudi obstaja naravna transformacija.
\end{primer}

\begin{primer}
Naj bo \cat{C} kategorija s produkti. Poglejmo si funktorja
\begin{align*}
\times : \cat{C}^2 \to \cat{C}, (ta vejica res tukaj ?)\\
\bar{\times} : \cat{C}^2 \to \cat{C},
\end{align*}
kjer je $\times$ običajen produkt in je $\bar{\times}$ definiran na objektih kot:
$$ A \bar{\times} B = B \times A$$
in na morfizmih
$$ \alpha \bar{\times} \beta = \beta \times \alpha.$$
Definiramo naravno transformacijo $t : \times \to \bar{\times}$ kot
$$t_{(A,B)}\langle a,b \rangle = \langle b,a \rangle,$$
za posplošene elemente $\langle a,b \rangle : Z \to A \times B$, tako
Da kvadrat:
$$ \begin{tikzcd}
A \times B \ar[d, "\alpha \times \beta"'] \ar[r, "t_{(A,B)}"] & B \times A \ar[d, "\beta \times \alpha"] \\
A' \times B' \ar[r, "t_{(A',B')}"'] & B' \times A'
\end{tikzcd} $$
komutira, pa mora za vsak posplošen element $\langle a,b \rangle$ veljati:
\begin{align*}
(\beta \times \alpha)(t_{(A,B)}\langle a,b \rangle ) &= (\beta \times \alpha)\langle b,a \rangle \\
& = \langle \beta b,\alpha a \rangle \\
& = t_{(A',B')}\langle \alpha a, \beta b \rangle \\
& = t_{(A', B')}(\alpha \times \beta)\langle a,b \rangle,
\end{align*}
kar pa velja po definiciji $t$. Torej je to res naravna transformacija.
\end{primer}

\begin{primer}
Naj bo $S$ stožec za diagram $D : \cat{J} \to \cat{C}$. V resnici je to samo objekt, za stožec so potrebni še morfizmi $s_i : S \to D_i$ za vsak $i \in \cat{J}$. Imamo torej družino morfizmov, za vsak objekt v neki kategoriji tako, da vse kar mora, komutira (verjetno potreba obrazložitev ?). To zelo spominja na definicijo naravne transformacije, le da imamo tu opravka z komutirajočimi trikotniki:
$$ \begin{tikzcd}
& S \ar[dl, "s_i"'] \ar[dr, "s_j"] & \\
D_i \ar[rr, "D_\alpha"'] & & D_j
\end{tikzcd} $$
Vendar vsak tak trikotnik lahko razširimo do kvadrata:

$$ \begin{tikzcd}
S \ar[d, "s_i"'] \ar[r, "1_S"] & S \ar[d, "s_j"]\\
D_i \ar[r, "D_\alpha"'] & D_j
\end{tikzcd} $$

Gre torej za naravno transformacijo iz konstantnega funktorja, ki pošlje vse objekte v $S$ in vse morfizme v $1_S$.
\end{primer}

\begin{definicija}
\emph{Funktorska kategorija} $\Fun(\cat{C}, \cat{D})$ ima za:
\begin{itemize}
\item objekte: funktorje $F : \cat{C} \to \cat{D}$
\item morfizme: naravne transformacije $\vartheta : F \to G$
\end{itemize}
Za vsak objekt $F$ ima naravna transformacija $1_F$ komponente
$$ (1_F)_C = 1_{FC} : FC \to FC$$
in kompozitum naravnih transformacij $F \xrightarrow{\vartheta} G \xrightarrow{\varphi} H$
ima komponente
$$ (\varphi \circ \vartheta)_C = \varphi_C \circ \vartheta_C.$$
\end{definicija}

\begin{primer}
Za vsako kategorijo $\cat{C}$ in končno kategorijo $\cat{1}$, je $\cat{C}^\cat{1} = \cat{C}$. Podobno za diskretno kategorijo z dvema objektoma $2 = \cat{Dis}(\set{0,1})$.
Objekti v tej kategoriji so funktorji
$$F,G : 2 \to \cat{C},$$
kjer označimo:
$$F(0) = F_0, \quad F(1) = F_1.$$
Morfizmi so naravne transformacije med temi funktorji
$$\alpha: F \to G.$$
Te naravne transformacije so pari morfizmov v $\cat{C}$, namreč
$$\alpha_0 : F_0 \to G_0, \alpha_1 : F_1 \to G_1.$$
Ker v $2$ ni dodatnih morfizmov, ne zahtevamo da kaj dodatnega komutira.
Če definiramo funktor $\cat{C}^2 \to \cat{C} \times \cat{C}$, ki deluje na objektih
$$F \mapsto \fprod{F_0,F_1}$$
in morfizmih
$$\alpha \mapsto (\fprod{F_0,F_1} \to \fprod{G_0,G_1}),$$
vidimo, da velja
$$\cat{C}^2 \cong \cat{C} \times \cat{C}.$$
Enako za vsako diskretno kategorijo $I$ velja:
$$\cat{C}^I \cong \prod_{i \in I}\cat{C}.$$

\end{primer}

\begin{primer} \label{grafi}
\emph{Usmerjen} graf sestoji iz vozlišč in povezav med njimi, kjer ima vsaka povezava "`rep"' in "`glavo"'. Grafe si lahko, tako kot kategorije, tudi narišemo:
%
$$
\begin{tikzcd}[arrow style=tikz, >=latex]
A \ar[r, "z"] \ar[rd, "u"] & B \\
C \ar[u, "x"] & D \ar[u, "y"'] \\
\end{tikzcd}
$$
Glavna razlika med kategorijo in grafom je, da v grafu ne obstaja nujno kompozitum povezav.
Graf torej sestoji iz dveh množic: $V$ množice vozlišč, $E$ množice povezav, in dveh funkcij med njima, $r : E \to V$ (rep) in $g : E \to V$ (glava). V $\cat{Set}$ je graf torej posebna konfiguracija objektov in morfizmov oblike:
%
$$
\begin{tikzcd}
E \ar[r, shift left, "r"] \ar[r, shift right, "g"'] & V \\
\end{tikzcd}
$$
%
Naj bo sedaj $\Gamma$ kategorija z dvema objektoma in dvema neidentitetnima morfizmoma, prikazana v sledečem diagramu:
%
$$
\begin{tikzcd}
\bullet \ar[r, shift left] \ar[r, shift right] & \bullet
\end{tikzcd}
$$
%
Kaj bi bila funktorska kategorija $\cat{Set}^\Gamma$? Objekti te kategorije so funktorji 
$$F,G : \Gamma \to \cat{Set},$$
ki sestojijo iz dveh množic in dveh funkcij med tema množicama. Morfizmi so naravne transformacije
$$\vartheta : F \to G$$
kjer morata oba kvadrata v diagramu:
%
$$
\begin{tikzcd}
F_0 \ar[r, "\vartheta_0"] \ar[d, shift left] \ar[d, shift right] & G_0 \ar[d, shift left] \ar[d, shift right] \\
F_1 \ar[r, "\vartheta_1"'] & G_1
\end{tikzcd}
$$
komutirati.(Mogoče nejasno kaj točno more komutirati)
V funktorjih $F,G$ lahko razpoznamo ravno usmerjene grafe, v naravnih transformacijah $\vartheta$ pa homomorfizme med grafi. To pa pomeni, da je
$$\cat{Set}^\Gamma = \cat{Grafi}.$$
(Tukaj bi mogoče mogli definirati kaj točno je $\cat{Grafi}$)
\end{primer}

\chapter{Pomembne trditve in definicije}


\begin{trditev} \label{konstrukcija limit}
Kategorija ime vse \emph{končne limite} natanko takrat, ko ima vse končne produkte in zožke.

To da ima kategorija končne limite pomeni, da ima vsak končni diagram $D : \cat{J} \to \cat{C}$ limito v $\cat{C}$.

\end{trditev}
\begin{dokaz}
Če ima kategorija vse limite, potem ima očitno tudi produkte in zožke, saj sta to posebna primera limit. Bolj zanimiv del te trditve je njen obrat.
Naj bo $D : \cat{J} \to \cat{C}$ diagram v $\cat{C}$. Iščemo objekt v $\cat{C}$, ki ima morfizme do vsakega izmed $D_i$. Kot prva ideja je to produkt 
$$\prod_{i \in \cat{J}}D_i,$$
ki ima projekcije do vsakega $D_i$. Na žalost tej morfizmi ne komutirajo glede na morfizme v $\cat{J}$. Radi bi še, da za vsak $\alpha : j \to k$ iz  $\cat{J}$ diagram:
%
$$ \begin{tikzcd}
& \prod_{i \in \cat{J}}D_i \ar[dl, "\pi_j"'] \ar[dr, "\pi_k"] & \\
D_j \ar[rr, "D_\alpha"'] && D_k
\end{tikzcd} $$
komutira. Za to si oglejmo produkt
$$ \prod_{\alpha \in Arr(\cat{J})} D_{cod(\alpha)} $$
po vseh morfizmih v $\cat{J}$ in dva posebna morfizma
$$ \begin{tikzcd}[sep=normal]
\prod_{i }D_i \ar[r, shift left, "\phi"] \ar[r, shift right, "\psi"'] & \prod_{\alpha } D_{cod(\alpha)},
\end{tikzcd} $$
ki ju definiramo glede na njun kompozitum s projekcijami $\pi_\alpha$ iz drugega produkta, specifično:
\begin{align*}
\pi_\alpha \circ \phi &= \phi_\alpha = \pi_{cod(\alpha)} \\
\pi_\alpha \circ \psi &= \psi_\alpha = D_\alpha \circ \pi_{dom(\alpha)}
\end{align*}
Da dobimo limito, vzamemo zožek teh dveh morfizmov:
$$ \begin{tikzcd}[sep=normal]
E \ar[r, "e"] & \prod_{i }D_i \ar[r, shift left, "\phi"] \ar[r, shift right, "\psi"'] & \prod_{\alpha } D_{cod(\alpha)}
\end{tikzcd} $$
in definiramo $e_i := \pi_i \circ r$. Da pokažemo, da je to res limita, vzamemo poljuben morfizem $c : C \to \prod_i D_i$ in pišemo $c = \langle c_i \rangle$, za $c_i = \pi_i \circ c$. Družina morfimov $(c_i : C \to D_i)_{i \in \cat{J}}$ je stožec nad $D$ natanko takrat, ko velja $\psi \circ c = \phi \circ c$, kajti
$$\phi \langle c_i \rangle = \psi \langle c_i \rangle,$$
natanko tedaj, ko je za vsak $\alpha$
$$(\pi_\alpha \circ \phi) \langle c_i \rangle = (\pi_\alpha \circ \psi) \langle c_i \rangle.$$
Velja pa
$$(\pi_\alpha \circ \phi) \langle c_i \rangle = \phi_\alpha \langle c_i \rangle = \pi_{cod(\alpha)}\langle c_i \rangle = c_{cod(\alpha)} $$
ter
$$(\pi_\alpha \circ \psi) \langle c_i \rangle = \psi_\alpha \langle c_i \rangle  = D_\alpha \circ \pi_{dom(\alpha)} \langle c_i \rangle = D_\alpha \circ c_{dom(\alpha)} $$
Sledi, da je $(E, e_i)$ res stožec in da vsak stožec $(c_i : C \to D_i)$ porodi morfizem $\langle c_i \rangle : C \to \prod_i D_i$, da velja $\phi\langle c_i \rangle = \psi \langle c_i \rangle$. Torej obstaja enolična faktorizacija $\langle c_i \rangle$ skozi $E$.

\end{dokaz}

\begin{posledica}
Kategorija ima vse končne kolimite natanko takrat, ko ima vse končne koprodukte in kozožke.
\end{posledica}
\begin{dokaz}
Dualnost.
\end{dokaz}

\begin{posledica}
Kategorija $\cat{Set}$ je kompletna in kokompletna.
\end{posledica}
\begin{dokaz}
Sledi iz dejstva, da ima $\cat{Set}$ produkte, koprodukte, zožke in kozožke, katerih obstoj je bil dokazan s konstrukcijo skozi primere. (citiranje primerov ?)
\end{dokaz}

\begin{posledica}
Kategorija $\cat{Set}$ je kartezično zaprta.
\end{posledica}
\begin{dokaz}
Tudi eksponente smo konstruirali eksplicitno v primeru.
\end{dokaz}
 
\begin{trditev} Predstavljivi funktor $\Hom_{\bf{C}}(C,-): \bf{C} \to \bf{Sets}$ ohranja vse limite.
\end{trditev}
\begin{dokaz}
Po trditvi \ref{konstrukcija limit} je potrebno pokazati le, da $\Hom$ funktor ohranja produkte in zožke:
%
\begin{itemize}
\item za končni objekt $1$ v $\cat{C}$ je 
$$\Hom(C,1) \cong \set{*} \cong 1$$
tudi končni objekt v $\cat{Set}.$(ali je tukaj res pika, kaj se potem naslednja točka začne z veliko začetnico?)
\item za binarni produkt $X \times Y$ v $\cat{C}$ definiramo morfizem 
$$h : \Hom(C, X \times Y) \to \Hom(C,X) \times \Hom(C,Y),$$
kot $h(f : C \to X \times Y) = \langle \pi_1 \circ f, \pi_2 \circ f \rangle$, ki po univerzalni lastnosti produkta določa enoličen izomorfizem.
\item podobno kot za binarne produkte naredimo za splošne produkte
$$\Hom(C, \prod_i X_i) \cong \prod_i \Hom(C, X_i)$$
\item naj bo sedaj
$$ \begin{tikzcd}[sep=normal]
E \ar[r, "e"] & X \ar[r, shift left, "f"] \ar[r, shift right, "g"'] & Y
\end{tikzcd} $$
zožek v $\cat{C}$, ki nam poda diagram
$$ \begin{tikzcd}[sep=normal]
\Hom(C,E) \ar[r, "e_*"] & \Hom(C,X) \ar[r, shift left, "f_*"] \ar[r, shift right, "g_*"'] & \Hom(C,Y)
\end{tikzcd} $$
v $\cat{Set}$. Da pokažemo, da je to zožek, vzemimo neki $h \in \Hom(C,X)$ z $f_*(h) = g_*(h)$. Potem velja $f \circ h = g \circ h$, torej obstaja enoličen $u : C \to E$, tako, da velja $e \circ u = h$. Torej je $e_* : \Hom(C,E) \to \Hom(C,X)$ res zožek $f_*$ in $g_*$.
\end{itemize}

\end{dokaz}

\begin{posledica}
Kontravariantni predstavljivi funktor $\Hom_\cat{C}(-,C) : \cat{C}^{op} \to \cat{Set}$ ohranja vse kolimite.
\end{posledica}

\begin{trditev}
$\Hom(A, C^B) \cong \Hom(A \times B, C).$
\end{trditev}
\begin{dokaz}
Sledi iz definicije eksponenta in transponiranja.
\end{dokaz}


\chapter{Yonedova lema}

V tem poglavju si bomo bolj podrobno pogledali posebne funktorske kategorije oblike
$$\predsnop{C},$$
kjer je $\cat{C}$ lokalno majhna kategorija. V taki kategoriji so objekti funktorji 
$$F,G : \cat{C} \to \cat{Set}$$
in morfizmi so naravne transformacije med takimi funktorji
$$\alpha, \beta : F \to G.$$ 
V vsaki taki kategoriji $\predsnop{C}$, lahko evaluiramo vsak komutirajoč(je to beseda ?) diagram:
%
$$
\begin{tikzcd}
P \ar[r, "\alpha"] \ar[rd, "\beta \circ \alpha"'] & Q \ar[d, "\beta"] \\
& R
\end{tikzcd}
$$
v nekem objektu $C$, da dobimo diagram v $\cat{Set}$:
%
$$
\begin{tikzcd}
PC \ar[r, "\alpha_C"] \ar[rd, "(\beta \circ \alpha)_C"'] & QC \ar[d, "\beta_C"] \\
& RC
\end{tikzcd}
$$
To pa pomeni, da imamo evaluacijski funktor
$$ev_C : \predsnop{C} \to \cat{Set}.$$
Prav tako, če imamo v $\cat{C}$ morfizem $f : D \to C$, zaradi naravnosti dobimo "`cilinder"' nad zgornjim diagramom.


Na take funktorske kategorije pa lahko gledamo še na drugačen način.
Če se spomnemo primera \ref{grafi} smo videli, da je za kategorijo $\Gamma$, funktorska kategorija $\cat{Set}^\Gamma$ enaka usmerjenim grafom. To nakazuje, da si za splošno kategorijo $\cat{C}$, kategorijo
$$\cat{Set}^\cat{C}$$
lahko predstavljamo kot posplošeno kategorijo strukturiranih množic in morfizmov, ki ohranjajo to strukturo med njimi.


\section{Yonedova vložitev}

V svetu matematike poznamo mnogo primerov, ko delamo z neko matematično strukturo in v njej rešujemo specifičen problem, kot na primer iskanje ničel polinomov v realnih številih. Včasih pa je mogoče strukturo v kateri rešujemo problem, razširiti do strukture, kjer problem postane lažje rešljiv. Tako v primeru realnih števil ugotovimo, da nekateri polinomi enostavno nimajo ničel v tej množici, a, ko realna števila razširimo do kompleksnih števil, kar naenkrat postanejo rešljive vse polinomske enačbe. Na analogen način bomo to storili s kategorijami, kjer bomo vzeli neko kategorijo $\cat{C}$, ki mogoče ne bo imela vseh lastnosti, ki bi si jih želeli, in to kategorijo vložili v "`lepšo"' kategorijo, na način, ki spominja na razširitev realnih v kompleksna števila. 


\begin{definicija} \emph{Yonedova vložitev} je funktor 
$$y : \bf{C} \to \bf{Sets}^{\bf{C}^{op}}$$
ki je na objektih $C \in \cat{C}$ definiran kot:
$$y(C) := \Hom_{\cat{C}}(-, C) : \cat{C}^{op} \to \cat{Set}$$ 
in na morfizmih $f : C \to D$
$$y(f) := \Hom(-,f) : \Hom(-,C) \to \Hom(-,D).$$
\end{definicija}

Funktorju pravimo \emph{vložitev}, če je zvest, poln, in injektiven na objektih. Kasneje bomo pokazali, da je funktor $y$ res vložitev. 

\section{Yonedova lema}

Profesor Nobuo Yoneda se je rodil 28 marca, leta 1930. Matematiko je študiral na univerzi v Tokiu. V času njegovega študija je Tokijško univerzo obiskal prof. Samuel Eilenberg in Yoneda je z njim potoval po Japonski kot vodič in prevajalec. Kasneje je pridobil Fulbrightovo štipendijo in obiskal Princeton, kjer je študiral pod Eilenbergom. Kmalu po tem, ko je Yoneda prispel v Princeton je Eilenberg odpotoval v Francijo, kar je po enem letu storil tudi Yoneda. V tem času se je v povezavi s knjigo, ki jo je pisal o teoriji kategorij, Saunders Mac Lane srečeval z ljudmi, ki so to temo poznali in na ta način prišel v stik z mladim Yonedo. Njun interviju se je začel v Caf'e at Gare du Nord in trajal vse do odhoda Yonedovega vlaka. Vsebino tega pogovora je Mac Lane poimenoval kot Yonedova lema.


Glavna ideja Yonedove leme je v tem, da je za opis kontravariantnih funktorjev v $\cat{Set}$ dovolj poznati le delovanje predstavljivih funktorjev, saj lahko z njimi izrazimo poljuben drug funktor, na naraven način. Poglejmo točno formulacijo.


\begin{izrek}[Yonedova lema]
Naj bo $\cat{C}$ lokalno majhna kategorija. Potem za vsak objekt $C \in \bf{C}$ in funktor $F : \cat{C}^{op} \to \cat{Set}$ velja
$$\Hom(yC,F) \cong FC$$
Ta izomorfizem je naraven tako v $C$ kot v $F$, kar pomeni, da za $f : C \to D$ naslednji diagram:
%
\begin{equation} \label{diag1}
\begin{tikzcd}[row sep=huge]
\Hom(yC,F) \arrow[r, "\cong"] \arrow[d, "{\Hom(yf,F)}"'] & FC \arrow[d, "F(f)"] \\
\Hom(yD,F) \arrow[r, "\cong"'] & FD
\end{tikzcd}
\end{equation}
%
komutira ter za naravno transformacijo $\varphi : F \to G$ naslednji diagram:
%
\begin{equation} \label{diag2}
\begin{tikzcd}[row sep=huge]
\Hom(yC,F) \arrow[r, "\cong"] \arrow[d, "{\Hom(yC,}\varphi\text{)}"'] & FC \arrow[d, "\varphi_C"] \\
\Hom(yC,G) \arrow[r, "\cong"']	&	GC
\end{tikzcd}
\end{equation}
tudi komutira.
%
\begin{opomba}
 $\Hom(yC,F)$ je množica naravnih transformacij med funktorjema $yC,F : \cat{C} \to \cat{Set}$, oziroma 
 $$\Hom(yC,F) = \Hom_{\cat{Set}^{\cat{C}^{op}}}(yC,F) = \mathrm{Nat}(yC,F).$$
\end{opomba}
%
\end{izrek}
\begin{dokaz}
Poglejmo, kaj mora veljati, za neko naravno transformacijo
$$\vartheta : yC \to F,$$
ki je v bistvu družina morfizmov 
$$(\vartheta_D : yC(D) \to F(D))_{D \in \bf{C}}.$$
Tej morfizmi morajo izpolnjevati naravnostni pogoj, da za vsak morfizem $f : D \to C$, naslednji diagram:
%
\[ \begin{tikzcd}[row sep=huge]
\Hom(C,C) \arrow[r, "\vartheta_C"] \arrow[d, "{\Hom(f,C)}"'] & FC \arrow[d, "\text{F(f)}"] \\
\Hom(D,C) \arrow[r, "\vartheta_D"'] & FD
\end{tikzcd} \]
komutira. Torej mora veljati 
$$(F(f) \circ \vartheta_C) (g) = (\vartheta_D \circ \Hom(f,C)) (g)$$
za vsak $g \in \Hom(C,C)$. En tak $g$ je identiteta na C, oziroma $1_C : C \to C$. Veljati mora zato enakost:

\begin{align} \label{eq1}
\begin{split}
\underline{(F(f) \circ \vartheta_C)(1_C)}& = (\vartheta_D \circ \Hom(f,C))(1_C) = \\
(\vartheta_D \circ \Hom(f,C))(1_C)& = (\vartheta_D \circ ( \_ \circ f ))(1_C) =
\vartheta_D \circ ( 1_C \circ f) = \underline{\vartheta_D(f)}
\end{split}
\end{align}
Vidimo torej, da je vrednost komponente za $\vartheta$ v $D$ določena že s tem, kam $\vartheta_C$ slika $1_C$.
Naj bo z 
$$\alpha_{C,F} : \Hom(yC,F) \to FC$$ 
označen iskani izomorfizem.
Definiramo torej naravno transformacijo $\vartheta \in \Hom(yC,F)$ kot
\begin{equation}
\alpha_{C,F}(\vartheta) := \vartheta_C(1_C)
\end{equation}
%
Označimo:
\begin{equation}
\boxed{\alpha_{C,F}(\vartheta) = \widehat{\vartheta}}
\end{equation} 
%
in za vsak element $a \in FC$ definirajmo naravno transformacijo $\vartheta_a \in \Hom(yC,F)$ po komponentah, in sicer:
\begin{align}
&(\vartheta_a)_D : yC(D) \to F(D) \\
&(\vartheta_a)_D(f : D \to C) := F(f)(a).
\end{align}
%
Označimo:
\begin{equation}
\boxed{\alpha^{-1}_{C,F}(a) = \widetilde{a}}
\end{equation}
%
Sedaj je potrebno preveriti, da tako definirana preslikava res ustreza pogojem izreka.
Najprej preverimo, da sta si predpisa vzajemno inverzna.
Torej, da za vsak $\vartheta \in \Hom(yC,F)$ velja: 
$$\widetilde{\widehat{\vartheta}} = \vartheta$$
in za vsak $a \in FC$ velja
$$\widehat{\widetilde{a}} = a.$$
\begin{itemize}
\item
$\widehat{\vartheta} = \vartheta_C(1_C) \in FC$.
Potem je 
$$\widetilde{\widehat{\vartheta}} = \eta_{\widehat{\vartheta}}.$$
Poglejmo kako ta naravna transformacija deluje na morfizmih. Naj bo $f : D \to C$. Velja:
\begin{align*}
(\eta_{\widehat{\vartheta}})_D : yC(&D) \to FD \\
&f \xmapsto{def} Ff(\widehat{\vartheta}) \\
\end{align*}
%
in zato je:
\begin{align*}
&F(f(\widehat{\vartheta})) = F(f(\vartheta_C(1_C))) = \vartheta_D(f) \\
\implies& \forall f \in \textbf{C}_1  : (\eta_{\widehat{\vartheta}})_D(f) = \vartheta_D(f) \\
\implies& \forall D \in \textbf{C} : (\eta_{\widehat{\vartheta}})_D = \vartheta_D \\
\implies& \eta_{\widehat{\vartheta}} = \vartheta
\implies \underline{\widetilde{\widehat{\vartheta}} = \vartheta}
\end{align*}
\item še v drugo smer. Najprej opazimo:
$$\widetilde{a} = \vartheta_a : yC \to F$$
Velja
\begin{align*}
(\vartheta_a)_D : yC(D) \to FD \\
f \mapsto F(f)(a) \\
\end{align*}
Torej je res:
$$\widehat{\widetilde{a}} = (\vartheta_a)_C(1_C) = F(1_C)(a) = 1_{F(C)}(a) = a.$$
\end{itemize}
%
Sedaj moramo še pokazati, da sta zadoščena naravna pogoja.
Naj bo $f : C \to D$ in naj bo $\vartheta \in \Hom(yC,F)$.
Pokazali bi radi:
\begin{equation}
(F(f) \circ \alpha_{C,F})(\vartheta) = (\alpha_{D,F} \circ Hom(yf,F)(\vartheta)
\end{equation}
Velja pa:
$$(F(f) \circ \alpha_{C,F}(\vartheta) = \underline{F(f(\vartheta_C(1_C)))}$$
in po drugi strani:
\begin{align*}
(\alpha_{D,F} \circ \Hom(yf,F))(\vartheta) = \alpha_{D,F}(\vartheta \circ yf) = \\
(\vartheta \circ yf)_D (1_D) = (\vartheta_D \circ yf_D)(1_D) = \\
\vartheta_D(f \circ 1_D) = \underline{\vartheta_D(f)}.
\end{align*}
Enakost (\ref{eq1}) pa nam pove ravno, da je $F(f(\vartheta_C(1_C))) = \vartheta_D(f)$
To pa pomeni, da diagram (\ref{diag1}) komutira.

Kaj pa drugi diagram? Naj bo $\varphi : F \to G$ naravna transformacija med funktorjema $F,G : \cat{C}^{op} \to \cat{Set}$.
Veljati mora enakost:
$$(\varphi_C \circ \alpha_{C,F})(\vartheta) = (\alpha_{C,G} \circ \Hom(yC,\varphi))(\vartheta).$$
Imamo:
$$(\varphi_C \circ \alpha_{C,F})(\vartheta) = \underline{\varphi_C(\vartheta_C(1_C))}$$
In za desno stran:
$$(\alpha_{C,G} \circ \Hom(yC,\varphi))(\vartheta) = \alpha_{C,G}(\varphi \circ \vartheta) = 
(\varphi \circ \vartheta)_C(1_C) = (\varphi_C \circ \vartheta_C)(1_C) = \underline{\varphi_C(\vartheta_C(1_C)}.$$
Torej diagram (\ref{diag2}) tudi komutira.

\end{dokaz}

\section{Posledice Yonedove leme}

Takoj dobimo posledico, ki upraviči poimenovanje funktorja $y$ kot vložitev
\begin{posledica} Funktor $y : \cat{C} \to \cat{Set}^{\cat{C}^{op}}$ je poln in zvest.
\end{posledica}
\begin{dokaz}
Funktor $F:\cat{C} \to \cat{D}$ je zvest, če je za vsaka $A,B \in \cat{C}$ funkcija $F_{A,B} : \Hom(A,B) \to \Hom(FA,FB)$ injektivna, in poln, če je surjektivna.
V našem primeru imamo za poljubna $A,B \in \cat{C}$ 
$$\Hom(yA,yB) \cong yB(A) \cong \Hom(A,B),$$
kjer prvi izomorfizem sledi iz Yonedove leme. Torej je ta inducirana funkcija za vsaka $A$ in $B$ bijekcija in je $y$ res zvest in poln.
\end{dokaz}
\begin{opomba}
Opazimo lahko, da je Yonedova vložitev $y : \cat{C} \to \cat{Set}^{\cat{C}^{op}}$ tudi injektivna na objektih, saj za poljubna $A,B \in \cat{C}$, za katera velja $yA = yB$, potem v posebnem primeru velja $yA(A) = yA(B)$ in $1_A \in \Hom(A,A) \implies 1_A \in \Hom(B,A) \implies B = A$.
\end{opomba}


Naslednja posledica nam poda način, kako nam Yonedova lema lahko olajša dokazovanje.

\begin{posledica} \emph{(Yonedov princip)}
Za poljubna objekta $A$ in $B$ v lokalno majhni kategoriji $\cat{C}$,
$$\text{iz} \quad yA \cong yB \quad\text{sledi}\quad A \cong B .$$
\end{posledica}
\begin{dokaz}
Za $F : \cat{C}^{op} \to \cat{Set}$ vzamemo kar $yB$ in dobimo: $Hom(yA,yB) \cong Hom(A,B)$. Ker pa bijekcija slika izomorfizem v izomorfizem ($y$ je namreč poln in zvest po prejšnji posledici), res velja $A \cong B$.
\end{dokaz}


Poglejmo si pogosto uporabo tega principa.

\begin{primer}
Radi pokazali, da v kartezično zaprti kategoriji $\cat{C}$ za poljubne $A,B,C \in \cat{C}$ velja 
$$(A^B)^C \cong A^{(B \times C)}.$$
Po Yonedovem principu je dovolj preveriti, da velja $y((A^B)^C) \cong y(A^{(B \times C)})$ ter, da je ta izomorfizem naraven. Torej vzamemo poljuben $X \in \cat{C}$ in računamo:
\begin{align*}
\Hom(X, (A^B)^C) \cong &\Hom(X \times C, A^B) \\
&\Hom(X \times C \times B, A) \\
&\Hom(X, A^{(B \times C)}) \\
\end{align*}
Prevereti je še potrebno, da so tej izomorfizmi naravni v $X$.
Naj bo torej $f : Y \to X$ v $ \cat{C}$. Situacija v $\cat{C}$ je naslednja:
%
\begin{equation}
\begin{tikzcd}
(A^B)^C & (A^B)^C \times C \ar[r, "\epsilon"] & C \\
X \ar[u, "g"] & X \times C \ar[u, "g \times 1_C"] \ar[ur, "\overline{g}"'] & \\
Y \ar[u, "f"] & Y \times C \ar[u, "f \times 1_C"] \ar[uur, bend right, "\overline{g \circ f}"'] & \\
\end{tikzcd}
\end{equation}
kjer je $\epsilon$ evaluacija, in $\overline{g}$ transponiranka $g$. Po enoličnosti transponiranja je $\overline{g \circ f}$ enolični morfizem, da ta diagram komutira za $(g \circ f) \times 1_C$.

\end{primer}


Sedaj si poglejmo, v kakem smislu je kategorija predsnopov $\predsnop{C}$ "`lepa"'. Nekaj o tem nam pove naslednja trditev.

\begin{trditev} \label{trditev 3.3}
Za vsako lokalno majhno kategorijo $\cat{C}$ je funktorska kategorija $\predsnop{C}$ kompletna. Za vsak objekt $C \in \cat{C}$, evaluacijski funktor 
$$ev_C : \predsnop{C} \to \cat{Set}$$
ohranja vse limite.
\end{trditev}
\begin{dokaz}
Naj bo $\cat{J}$ majhna indeksna kategorija in $D : \cat{J} \to \cat{Set}^{\cat{C}^{op}}$ diagram oblike $\cat{J}$. Če naj bi taka limita obstajala, bi to bil objekt v $\cat{Set}^{\cat{C}^{op}}$, oziroma funktor 
$$(\lim_{j \in J} D_j): \cat{C}^{op} \to \cat{Set}$$
po Yonedovi lemi pa bi za tak funktor moralo veljati
$$(\lim_{j \in J} D_j)(C) \cong \Hom(yC, \lim D_j),$$
ker pa vemo, da predstavljivi funktorji ohranjajo limite velja
$$\Hom(yC, \lim D_j) \cong \lim_{j \in J} Hom(yC, D_j).$$
Z ponovno uporavo Yonedove leme dobimo
$$\lim_{j \in J} Hom(yC, D_j) \cong \lim_{j \in J} D_j(C).$$
Torej limito v $\predsnop{C}$ definiramo kar kot $\lim_{j \in J} (D_j(C))$, oziroma po točkah.
%
Preostane še poiskati oziroma definirati stožec nad tem objektom in pokazati, da je končen med stožci nad $D$.


\end{dokaz}

Povemo pa lahko še več. Naslednji izrek, všasih imenovan tudi \emph{gostotni izrek}, je v nekem smislu dual Yonedove leme.

\begin{trditev}
\label{density theorem}
Za vsako majhno kategorijo $\cat{C}$, se vsak objekt $P$ funktorske kategorije $\predsnop{C}$ da izraziti, kot kolimito predstavljivih funktorjev iz neke indeksne kategorije $\cat{J}$.
$$ P \cong \colim_{j \in J} yC_j.$$
Bolj natančno, obstaja kanonična izbira indeksne kategorije $\cat{J}$ in funktorja $\pi : \cat{J} \to \cat{C}$, tako da obstaja naravni izomorfizem $\colim y \circ \pi \cong P$.
\end{trditev}
\begin{dokaz}
Naj bo torej $P : \cat{C}^{op} \to \cat{Set}$ objekt v kategoriji predsnopov nad $\cat{C}$. Za indeksno kategorijo bomo definirali tako imenovano \emph{kategorijo elementov P}, ki se jo označuje z
$$\int_{\cat{C}}P$$
in ima za:
\begin{itemize}
\item objekte: pare $(x,C)$, kjer je $C \in \cat{C}$ in $x \in P(C)$.
\item morfizme: trojice $(h, (x',C'), (x,C))$, kjer je $h : C' \to C$ morfizem v $\cat{C}$, tako da velja $P(h(x)) = x'$. Zaradi priročnosti te morfizme ponavadi označujemo kar s $h : (x', C') \to (x,C)$ in se zavedamo, da morajo zadoščati pogoju.
Identiteta na $(x,C)$ je kar podedovana identiteta na $C$, kajti funktorji ohranjajo indetitete. Kompozitum dveh takih morfizmov je spet tak morfizem, kajti za $h : (x'', C'') \to (x', C')$ in $k : (x', C') \to (x,C)$ velja $P(k \circ h)(x) = (P(h) \circ P(k))(x) = P(h(x')) = x''$.
\end{itemize}
Definiramo funktor $\pi : \int_{\cat{C}}P \to \cat{C}$, kot $\pi(x,C) = C$ in trdimo, da velja 
$$\colim y\circ \pi (x,C) \cong P.$$
Da bo to res kolimita, potrebujemo morfizme $y\circ \pi (x,C) \to P$, za vsak $(x,C) \in \int_{\cat{C}}P$. Po Yonedovi lemi imamo bijekcijo
$$x \in PC \leftrightsquigarrow x:yC \to P,$$
ki je naravna v $C$, torej za $h: (x',C') \to (x,C)$ naslednji diagram:
$$ \begin{tikzcd}
yC' \ar[rr, "yh"] \ar[rd, "x'"'] & & yC \ar[ld, "x"] \\
& P &
\end{tikzcd} $$
komutira. Za komponente kostožca nad $y \circ \pi$ vzamemo torej naravne transformacije $x : yC \to P$. Da vidimo, da je to res kolimita, denimo, da obstaja neki drug stožec $y \circ \pi \to Q$, s komponentami $\vartheta_{(x,C)} : yC \to Q$. Iščemo torej tako enolično naravno transformacijo, da naslednji diagram komutira.
%
$$\begin{tikzcd}
yC' \ar[rr, "yh"] \ar[ddr, bend right, "\vartheta_{(x',C')}"'] \ar[rd, "x'"'] & & yC \ar[ld, "x"] \ar[ddl, bend left, "\vartheta_{(x,C)}"] \\
& P \ar[d, dashed, "\vartheta"] & \\
& Q &
\end{tikzcd} $$
Ponovno lahko po Yonedovi lemi identificiramo 
$$\vartheta_{(x,C)}: yC \to Q \leftrightsquigarrow \vartheta_{(x,C)} \in QC$$
in za komponente naravne transformacije $\vartheta : P \to Q$ vzamemo kar funkcijo, ki jo inducira ta bijekcija, torej 
$$\vartheta_C : PC \to QC, \quad \vartheta_C(x) = \vartheta_{(x,C)},$$
ker je zgornji izomorfizem naraven v C, to implicira komutativnost diagrama. Preveriti je potrebno še enoličnost $\vartheta$. Naj bo torej $\psi : P \to Q$ tak, da stranice trikotnika komutirajo. Potem velja 
$$\psi \circ x = \vartheta_{(x,C)} = \vartheta \circ x.$$
\end{dokaz}

Kako pa je z eksponenti v $\cat{Set}^{\cat{C}^{op}}$? Če sta $Q,P$ funktorja v $\predsnop{C}$, kako bi definirali eksponentni objekt $Q^P$? Po Yonedovi lemi bi moralo veljati $Q^P(C) \cong \Hom(yC, Q^P)$. Če pa želimo, da ta objekt izpolnjuje lastnost eksponenta, mora veljati
\begin{equation} \label{eq eksp}
\Hom(yC,Q^P) \cong \Hom(yC \times P, Q).
\end{equation}
po lastnosti transponiranja. Ker pa $\predsnop{C}$ ima produkte, kajti ima vse končne limite, množica $\Hom(yC \times P, Q)$ obstaja. Na ta način tudi definiramo eksponent.

\begin{trditev} \label{trditev 3.6}
Za lokalno majhno kategorijo $\cat{C}$, za vse $X,P,Q$ v $\predsnop{C}$ obstaja izomorfizem
$$\Hom(X,Q^P) \cong \Hom(X \times P, Q),$$
ki je naraven v $X$.
\end{trditev}

Najprej potrebujemo še dodatno lemo
\begin{lema}
Za vsako majhno indeksno katgorijo $\cat{J}$, funktor $A : \cat{C} \to \predsnop{C}$ in diagram $B \in \predsnop{C}$, obstaja naravni izomorfizem
\begin{equation} \label{eq 3.11}
(\underset{j \in J}{\colim}A_j) \times B \cong \underset{j \in J}{\colim}(A_j \times B).
\end{equation}
\end{lema}
\begin{dokaz}
Naj bodo 
$$(\vartheta_i : A_i \to \underset{j \in J}\colim A_j)_{i \in J}$$
komponente kostožca, za kolimito nad $A$. To komponiramo s funktorjem $$- \times B : \predsnop{C} \to \predsnop{C0},$$
da dobimo komponente kostožca:
$$-\times B(\vartheta_i)=: \vartheta_i \times B : A_i \times B \to (\underset{j \in J}\colim A_j) \times B, \quad i \in J.$$
Obstaja torej enoličen morfizem iz kolimite nad diagramom $(- \times B) \circ A$, ki ga označimo z $\vartheta :  \underset{j \in J}\colim (A_j \times B) \to (\underset{j \in J}\colim A_j) \times B$, tako da vse ustrezno v diagramu:
%
$$
\begin{tikzcd}
&  \underset{j \in J}\colim (A_j \times B) \ar[dd, dashed, "\vartheta"] & \\
A_i \times B \ar[ur] \ar[dr, "\vartheta_i \times B"'] & & A_k \times B \ar[ul] \ar[dl, "\vartheta_k \times B"] \\
&  (\underset{j \in J}\colim A_j) \times B &
\end{tikzcd}
$$
komutira. Radi bi pokazali, da je $\vartheta$ naravni izomorfizem. Zadošča pokazati, da so vse komponente
$$(\vartheta_C : \underset{j \in J}\colim (A_j \times B)(C) \to (\underset{j \in J}\colim A_j) \times B(C))_{C \in \cat{C}}$$
izomorfizmi v $\cat{Set}$, torej zadošča pokazati, da to velja za vse točke, kar pomeni, da lahko pokažemo (\ref{eq 3.11}), če predpostavimo, da so $A_i$ in $B$ množice. Za poljubno množico $X$ pa velja:
%
\begin{align*}
\Hom(\colim (A_j \times B), X) &\cong \colim\Hom(A_j \times B, X) \\
&\cong \colim\Hom(A_j, X^B) \\
&\cong \Hom(\colim A_j, X^B) \\
&\cong \Hom((\colim A_j) \times B, X). \\
\end{align*}
Tej izomorfizmi so jasno naravni v $X$, torej lema sledi iz Yonedove leme.(mogoče ni tako jasno ?)
\end{dokaz}


\begin{dokaz}(Dokaz trditve)
Po (\ref{density theorem}) velja, da obstaja indeksna kategorija $\cat{J}$, izbrana na kanoničen način, da je
$$ X \cong \underset{j \in J}{\colim}\> yCj.$$
Velja torej:

\begin{align*}
\Hom(X, Q^P) &\cong \Hom(\colim yC_j, Q^P) & \\
&\cong \colim\Hom(yC_j, Q^P) & (\Hom \text{ ohranja kolimite}) \\
&\cong \colim Q^P(C_j) & (\text{Yoneda}) \\
&\cong \colim \Hom(yC_j \times P, Q) & (\text{po \ref{eq eksp}}) \\
&\cong \Hom(\colim(yC_j \times P), Q) & \\
&\cong \Hom((\colim(yC_j) \times P, Q) & (\text{po lemi}) \\
&\cong \Hom(X \times P, Q) \\
\end{align*}
Tej izomorfizmi so očitno naravni v $X$, torej trditev sledi iz Yonedove leme.
\end{dokaz}

Dobljeno združimo v izrek.

\begin{izrek}
Za vsako majhno kategorijo $\cat{C}$, je kategorija predsnopov $\predsnop{C}$ kartezično zaprta. Prav tako, Yonedova vložitev
$$y : \cat{C} \to \predsnop{C}$$
ohranja vse produkte in eksponente v $\cat{C}$.
\end{izrek}
\begin{dokaz}
Po trditvi (\ref{trditev 3.3}) vemo, da ima $\predsnop{C}$ vse končne produkte. Če definiramo eksponente kot zgoraj, nam trditev (\ref{trditev 3.6}) pove, da ima tudi eksponente.
Pokazati je potrebno še drugi del izreka torej, da Yonedova vložitev ohranja produkte in eksponente. Na to noto denimo, da je $A \times B$ produkt v $\cat{C}$.
Naj bo $X,Y \in \cat{C}$ in $f : Y \to X$. Velja
$$ \Hom(X, A\times B) \cong \Hom(X, A) \times \Hom(X,B),$$
ki je seveda naraven v $X$ s postkompozicijo z $f$.(ali je potrebno definirati pojem postokompozicije?)
Za eksponente recimo, da imamo v $\cat{C}$ eksponent $Q^P$, za objekta $P,Q \in \cat{C}$. Po naši definiciji eksponentov v $\predsnop{C}$ velja:
\begin{align*}
y(Q)^{y(P)}(C) & \cong \Hom(yC \times yP, yQ) \\
&\cong \Hom(y(C \times P), yQ) \\
&\cong yQ(C \times P) \\
&\cong \Hom(C \times P, Q) \\
&\cong \Hom(C, Q^P) \cong y(Q^P)(C)
\end{align*}
in tej izomorfizmi so jasno naravni v $C$.
\end{dokaz}

Vidimo torej, da je kategorija $\predsnop{C}$ lepa kategorija v smislu, da je kartezično zaprta, če nas torej zanima neko dejstvo o kategoriji $\cat{C}$, lahko to kategorijo vložimo v kategorijo predsnopov, kjer lahko mogoče z njo lažje operiramo, saj lahko poljubno vzamemo(definiramo, uporabljamo ?) produkte, eksponente, itd.


Poglejmo si primer take vložitve.

\begin{primer}
Naj bo $\mathbf{\Delta}$ kategorija vseh končnih ordinalnih števil. Objekti so množice $[0],[1],[2],\ldots$, kjer je:
$$[0] = \emptyset \quad \text{in} \quad [n] = \set{0,1,\ldots,n}$$
in morfizmi so vse monotone funkcije med temi množicami.
Funktorsko kategorijo $\mathbf{\Delta}$ se včasih označuje tudi kot $\cat{sSet}$.
\begin{definicija}
\emph{Simplicirana množica}(pravilna terminologija ?) je funktor $\mathbf{\Delta}^{op} \to \cat{Set}$.

\end{definicija}
Naj bo torej
$$X : \mathbf{\Delta}^{op} \to \cat{Set}$$
simplicirana množica. Množico $X([n])$ se standardno označuje z $X_n$ in njene elemente imenuje \emph{n-simpleksi}. 
V kategoriji $\mathbf{\Delta}$ imamo posebne morfizme, ki generirajo vse morfizme v tej kategoriji. Za vsak $n \geq 0$ obstaja $n+1$ injekcij 
$$d^i : [n-1] \to [n]$$ 
imenovanih \emph{koobrazi} in $n+1$ surjekcij 
$$s^i : [n+1] \to [n]$$
imenovanih \emph{kodegeneracije}, definiriane na sledeči način:
%
$$d^i(k) = \begin{cases}
k, & k < i \\
k+1, & k \geq i 
\end{cases} $$
$$s^i(k) = \begin{cases}
k, & k \leq i \\
k-1, & k > i
\end{cases} $$
Tej morfizmi zadoščajo sledečim relacijam:
\begin{align*}
d^j d^i = d^i d^{j-1}, & \quad i < j \\
s^j s^i = s^i s^{j+1}, & \quad i \leq j 
\end{align*}
$$ s^j d^i = \begin{cases}
1, & i = j, j+1 \\
d^i s^{j-1}, & i < j \\
d^{i-1} s^j, & i > j+1
\end{cases} $$
%
Brez dokaza lahko povemo, da je mogoče vsak morfizem v $\mathbf{\Delta}$ izraziti kot kompozitum koobrazov in kodegeneracij. Če dodamo še dodatne zahteve o vrstnem redu, v katerem se lahko pojavijo, je tak zapis tudi enoličen.
%
Če je $X$ simplicirana množica, označujemo funkcije
\begin{align*}
d_i = X(d^i) : X_n \to X_{n-1} & & 0 \leq i \leq n \\
s_i = X(s^i) : X_n \to X_{n+1} & & 0 \leq i \leq n \\
\end{align*}
in jih imenujemo \emph{obrazi} in \emph{degeneracije}. Relacije med $d_i$ in $s_i$ bodo dualne tistim zgoraj.
Vsakemu $x \in X_n$ degeneracije priredijo $n+1$ $(n+1)$-simpleksov $s_0(x), s_1(x), \ldots , s_n(x)$ iz  $X_{n+1}$. Pravimo, da je simplex $x \in X_n$ \emph{degeneriran}, če je slika kake degeneracije $s_i$, sicer pravimo, da je \emph{ne-degeneriran}.
%
Najbolj enostaven primer simplicirane množice so tako imenovani \emph{standardni simpleksi} $\Delta^n$, ki so definirani za vsak objekt $[n] \in \mathbf{\Delta}$, kot:
$$\Delta^n := y([n]) = \Hom_{\mathbf{\Delta}}(-,[n]).$$
K-simpleksi v $\Delta^n$ so 
$$\Delta_k^n := \Hom([k], [n]).$$
Obrazi in degeneracije so podani s prekompozicijo v $\mathbf{\Delta}$ z $d^i$ in $s^i$.
\begin{align*}
d_i : \Delta_k^n \to \Delta_{k-1}^n & & s_i : \Delta_k^n \to \Delta_{k+1}^n \\
f \mapsto f \circ d^i & & f \mapsto f \circ s^i \\
\end{align*}
Simplicirana množica $\Delta^n$ ima enoličen ne-degeneiran n-simples, ki ustreza identiteti na $[n]$. Yonedova lema nam med drugim pove, da so funkcije med standardnimi simpleksi $f : \Delta^n \to \Delta^m$ v bijektivni korespondenci z morfizmi $f : [n] \to [m]$ v $\mathbf{\Delta}$. Yonedova lema nam pa pove tudi, da za vsako simplicirano množico $X$, obstaja naravna bijekcija, med n-simpleksi v $X$ in naravnimi transformacijami $\Delta^n \to X_n$ v $\cat{sSet}$. Bolj eksplicitno, n-simpleks $x \in X_n$ lahko obravnavamo kot morfizem $x : \Delta^n \to X$, ki pošlje enolični ne-degenerirani n-simplex v $\Delta^n$, v $x$. (mogoče malo nejasno napisano)
Simplicirani množici $X$ lahko konstruiramo asociirano kategorijo elementov
$$\int_{\mathbf{\Delta}}X,$$
ki se jo imenuje \emph{kategorija simpleksov}. Po trditvi (\ref{density theorem}) vemo, da lahko vsak tak $X$ izrazimo, kot kolimito standardnih simpleksov, indeksirano s svojo kategorijo elementov.

$$X \cong \underset{x \in X_n}{\colim}\Delta^n$$


Podajmo sedaj primer simplicirane množice. Naj bo $\cat{C}$ poljubna majhna kategorija. Definirajmo \emph{živec kategorije} $\cat{C}$, kot simplicirano množico $N\cat{C}$, na sledeči način:

\begin{enumerate}[label={}]
\item $N\cat{C}_0 = \set{\text{objekti v } \cat{C}}$
\item $N\cat{C}_1 = \set{\text{morfizmi v } \cat{C}}$
\item $N\cat{C}_2 = \set{\text{pari morfizmov } \rightarrow\rightarrow \text{v } \cat{C}}$ \\ % se opravičujem profesor, sem moral malo pohekati, da je ratal lep izpis
\vdots
\item $N\cat{C}_n = \set{\text{nizi n morfizmov } \rightarrow\rightarrow\ldots\rightarrow \text{v } \cat{C}}$
\end{enumerate}
Degeneracije $s_i:N\cat{C}_n \to N\cat{C}_{n+1}$ delujejo tako, da vzamejo niz n morfizmov
$$ c_0 \rightarrow c_1 \rightarrow \ldots \rightarrow c_i \rightarrow \ldots \rightarrow c_n$$
in na $i$-tem mestu vstavijo identiteto na $c_i$. Obraz $d_i : N\cat{C}_n \to N\cat{C}_{n-1}$ komponira $i$-t in $i+1$ morfizem, če je $0 < i < n$, oziroma izpusti prvi ali zadnji morfizem za, $i = 0$ ali $i = n$.
Potrebno bi bilo preveriti, da na ta način definirane funkcije res zadoščajo relacijam za simplicirano množico. Ker je bil namen tega primera ilustracija v kakih situacija lahko nastopa Yonedova lema, tega tukaj ne bomo storili.
\end{primer}

\begin{thebibliography}{9}

\bibitem{category theory}
Steve Awodey (2010)
\textit{Category Theory},
Oxford: Oxford University Press.

\bibitem{simplicial sets}
Emily Riehl
\textit{A leisurely introduction to simplicial sets}
\\\texttt{http://www.math.jhu.edu/~eriehl/ssets.pdf}
\end{thebibliography}

\end{document}
